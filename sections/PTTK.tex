\chapter{Phân tích thiết kế hệ thống}

\section{Sơ đồ Use Case - Tác nhân và hành động}

\subsection{Sơ đồ Use Case tổng quát}
\begin{figure}[H]
    \centering
    \includegraphics[scale=0.4]{images/Sơ đồ usecase tổng quát.png}
    \caption{Sơ đồ Use Case tổng quát}
\end{figure}

\subsection{Usecase đăng ký}
\begin{figure}[H]
    \centering
    \includegraphics[scale=0.4]{images/usecase đăng ký.png}
    \caption{Sơ đồ usecase đăng ký}
\end{figure}
\begin{longtable}{|m{4cm}|m{10cm}|}
    \hline
    \textbf{Tên Use Case} & \textbf{Đăng ký} \\
    \hline
    Tác nhân & Khách hàng \\
    \hline
    Mô tả & Khách hàng sử dụng chức năng này để đăng ký tài khoản phục vụ cho đăng nhập vào hệ thống \\
    \hline
    Điều kiện kích hoạt & Chọn chức năng đăng ký ở hệ thống  \\
    \hline
    Tiền điệu kiện & Tác nhân chưa đăng nhập và chưa có tài khoản ở hệ thống \\
    \hline
    Hậu diều kiện & 
    \begin{itemize}
        \item Nếu thành công: Hệ thống sẽ thông báo đăng ký tài khoản thành công. Người dùng được chuyển hướng đén trang đăng nhập.
        \item Nếu thất bại: Hệ thống sẽ đưa ra thông báo chi tiết lỗi và yêu cầu nhập lại thông tin.
    \end{itemize} \\
    \hline
    Luồng sự kiện chính & 
    \begin{enumerate}[label=\arabic{*}.]
        \item Hệ thống hiển thị giao diện đăng nhập
        \item Người dùng bấm vào nút đăng ký
        \item Hệ thống hiển thị giao diện đăng ký
        \item Người dùng nhập tài khoản bao gồm email mật khẩu và xác nhận mật khẩu
        \item Hệ thống kiểm tra thông tin đăng ký
        \item Hiển thị giao diện xác nhận email
        \item Người dùng nhập mã otp được gửi về email
        \item Chuyển hướng đến trang đăng nhập
        \item Kết thúc usecase
    \end{enumerate} \\
    \hline
    Luồng sự kiện phụ & 
        \begin{enumerate}[label=\textbf{\arabic*.}, leftmargin=1.2em]
        \item \textbf{Người dùng hủy đăng ký}
            \begin{enumerate}[label*=\arabic*.]
            \item Người dùng chọn “Hủy” để rời trang đăng ký.
            \item Hệ thống hiển thị hộp thoại xác nhận hủy.
            \item Nếu xác nhận: hủy phiên đăng ký tạm, xóa dữ liệu chưa lưu, không tạo tài khoản.
            \item Hệ thống chuyển hướng về trang đăng nhập.
            \item Kết thúc usecase.
            \end{enumerate}
        \item \textbf{Email đã tồn tại} (phát sinh ở bước 5 luồng chính):
            \begin{enumerate}[label*=\arabic*.]
            \item Hệ thống báo “Email đã được sử dụng”.
            \item Yêu cầu người dùng đăng nhập hoặc chọn “Quên mật khẩu”.
            \item Kết thúc usecase.
            \end{enumerate}

        \item \textbf{Mã xác thực email hết hạn} (ở bước 7 luồng chính):
            \begin{enumerate}[label*=\arabic*.]
            \item Hiển thị thông báo “Mã hết hạn”.
            \item Cung cấp nút “Gửi lại mã”.
            \item Khi gửi lại thành công, quay lại luồng chính tại bước 6.
            \item Kết thúc usecase.
            \end{enumerate}
        \end{enumerate} \\
    \hline
    Các yêu cầu đặc biệt & Không có \\
    \hline
\end{longtable}
\subsection{Usecase đăng nhập}
\begin{figure}[H]
    \centering
    \includegraphics[scale=0.4]{images/usecase đăng nhập.png}
    \caption{Sơ đồ usecase đăng nhập}
\end{figure}

\begin{longtable}{|m{4cm}|m{10cm}|}
    \hline
    \textbf{Tên Use Case} & \textbf{Đăng nhập} \\
    \hline
    Tác nhân & Người dùng \\
    \hline
    Mô tả & Tác nhân sử dụng usecase này để đăng nhập vào hệ thống\\
    \hline
    Điều kiện kích hoạt & Chọn chức năng đăng nhập ở hệ thống  \\
    \hline
    Tiền điệu kiện & Tác nhân chưa đăng nhập vào hệ thống \\
    \hline
    Hậu diều kiện & 
    \begin{itemize}
        \item Nếu thành công: Hệ thống sẽ hiển thị giao diện chính. Người dùng có thể thực hiện các chức năng theo đúng quyền hạn của mình
        \item Nếu thất bại: Hệ thống sẽ đưa ra thông báo "Thông tin đăng nhập không hợp lệ" và yêu cầu đăng nhập lại.
    \end{itemize} \\
    \hline
    Luồng sự kiện chính & 
    \begin{enumerate}[label=\arabic{*}.]
        \item Hệ thống hiển thị giao diện đăng nhập.
        \item Người dùng nhập tài khoản bao gồm tên đăng nhập và mật khẩu.
        \item Hệ thống kiểm tra và xác nhận thông tin đăng nhập.
        \item Hiển thị giao diện chính tùy vào vai trò của tài khoản đăng nhập là người dùng hay admin.
        \item Kết thúc usecase.
    \end{enumerate} \\
    \hline
    Luồng sự kiện phụ & 
        \begin{enumerate}[label=\textbf{\arabic*.}, leftmargin=1.2em]
        \item \textbf{Thông tin email hoặc mật khẩu sai}
            \begin{enumerate}[label*=\arabic*.]
            \item Người dùng nhập thông tin sai
            \item Hệ thống báo lỗi
            \item Kết thúc Use Case
            \end{enumerate}
        \end{enumerate} \\
    \hline
    Các yêu cầu đặc biệt & Không có \\
    \hline
\end{longtable}

\subsection{Usecase đăng xuất}
\begin{figure}[H]
    \centering
    \includegraphics[scale=0.6]{images/usecase đăng xuất.png}
    \caption{Sơ đồ usecase đăng xuất}
\end{figure}

\begin{longtable}{|m{4cm}|m{10cm}|}
    \hline
    \textbf{Tên Use Case} & \textbf{Đăng xuất} \\
    \hline
    Tác nhân & Người dùng \\
    \hline
    Mô tả & Tác nhân sử dụng usecase này để đăng xuất khỏi hệ thống\\
    \hline
    Điều kiện kích hoạt & Chọn chức năng đăng xuất ở hệ thống  \\
    \hline
    Tiền điệu kiện & Tác nhân đã đăng nhập vào hệ thống \\
    \hline
    Hậu diều kiện & 
    \begin{itemize}
        \item Nếu thành công: Hệ thống sẽ hiển thị giao diện khi chưa đăng nhập.
        \item Nếu thất bại: Hệ thống sẽ đưa ra thông báo lỗi.
    \end{itemize} \\
    \hline
    Luồng sự kiện chính & 
    \begin{enumerate}[label=\arabic{*}.]
        \item Hệ thống hiển thị giao diện đăng xuất.
        \item Người dùng chọn đăng xuất.
        \item Hệ thống tiến hành đăng xuất tài khoản người dùng.
        \item Hiển thị giao diện khi chưa đăng nhập.
        \item Kết thúc usecase.
    \end{enumerate} \\
    \hline
    Luồng sự kiện phụ & 
        \begin{enumerate}[label=\textbf{\arabic*.}, leftmargin=1.2em]
        \item \textbf{Đăng xuát thất bại}
            \begin{enumerate}[label*=\arabic*.]
            \item Người dùng đăng xuất thất bại
            \item Hệ thống báo lỗi
            \item Kết thúc Use Case
            \end{enumerate}
        \end{enumerate} \\
    \hline
    Các yêu cầu đặc biệt & Không có \\
    \hline
\end{longtable}

\subsection{Usecase đổi mật khẩu}
\begin{figure}[H]
    \centering
    \includegraphics[scale=0.6]{images/usecase đổi mật khẩu.png}
    \caption{Sơ đồ usecase đổi mật khẩu}
\end{figure}

\begin{longtable}{|m{4cm}|m{10cm}|}
    \hline
    \textbf{Tên Use Case} & \textbf{Đổi mật khẩu} \\
    \hline
    Tác nhân & Người dùng \\
    \hline
    Mô tả & Tác nhân sử dụng usecase này để thay đổi mật khẩu của tài khoản\\
    \hline
    Điều kiện kích hoạt & Chọn chức năng đổi mật khẩu ở khu vực tài khoản  \\
    \hline
    Tiền điệu kiện & Tác nhân đã đăng nhập vào hệ thống \\
    \hline
    Hậu diều kiện & 
    \begin{itemize}
        \item Nếu thành công: Hệ thống sẽ đăng xuất tài khoản người dùng và chuyển hướng đến trang đăng nhập.
        \item Nếu thất bại: Hệ thống sẽ đưa ra thông báo lỗi.
    \end{itemize} \\
    \hline
    Luồng sự kiện chính & 
    \begin{enumerate}[label=\arabic{*}.]
        \item Hệ thống hiển thị giao diện quản lý tài khoản.
        \item Người dùng chọn đổi mật khẩu.
        \item Hệ thống hiển thị giao diện đổi mật khẩu.
        \item Người dùng nhập mật khẩu cũ, mật khẩu mới và xác nhận mật khẩu mới.
        \item Hệ thống kiểm tra và xác nhận thông tin vừa nhập.
        \item Hiển thị thông báo đổi mật khẩu thành công và lưu thông tin mật khẩu mới vào cơ sở dữ liệu.
        \item  Hệ thống đăng xuất người dùng và chuyển hướng đến trang đăng nhập.
        \item Kết thúc usecase.
    \end{enumerate} \\
    \hline
    Luồng sự kiện phụ & 
        \begin{enumerate}[label=\textbf{\arabic*.}, leftmargin=1.2em]
        \item \textbf{Người dùng hủy yêu cầu đổi mật khẩu}
            \begin{enumerate}[label*=\arabic*.]
            \item Người dùng hủy yêu cầu đổi mật khẩu.
            \item Hệ thống bỏ qua và trở về giao diện quản lý tài khoản
            \item Kết thúc Use Case
            \end{enumerate}
        \item \textbf{Thông tin mật khẩu không hợp lệ}
            \begin{enumerate}[label*=\arabic*.]
            \item Người dùng nhập thông tin mật khẩu không hợp lệ
            \item Hệ thống thông báo lỗi
            \item Kết thúc Use Case
            \end{enumerate}
        \end{enumerate} \\
    \hline
    Các yêu cầu đặc biệt & Không có \\
    \hline
\end{longtable}


\subsection{Usecase chỉnh sửa thông tin tài khoản}
\begin{figure}[H]
    \centering
    \includegraphics[scale=0.6]{images/usecase chỉnh sửa thông tin.png}
    \caption{Sơ đồ usecase chỉnh sửa thông tin}
\end{figure}

\begin{longtable}{|m{4cm}|m{10cm}|}
    \hline
    \textbf{Tên Use Case} & \textbf{Chỉnh sửa thông tin} \\
    \hline
    Tác nhân & Người dùng \\
    \hline
    Mô tả & Tác nhân sử dụng usecase này để thay đổi thông tin của tài khoản\\
    \hline
    Điều kiện kích hoạt & Chọn chức năng chỉnh sửa ở khu vực quản lý tài khoản  \\
    \hline
    Tiền điệu kiện & Tác nhân đã đăng nhập vào hệ thống \\
    \hline
    Hậu diều kiện & 
    \begin{itemize}
        \item Nếu thành công: Hệ thống sẽ thông báo thành công và chuyển hướng đến trang quản lý tài khoản.
        \item Nếu thất bại: Hệ thống sẽ đưa ra thông báo lỗi.
    \end{itemize} \\
    \hline
    Luồng sự kiện chính & 
    \begin{enumerate}[label=\arabic{*}.]
        \item Hệ thống hiển thị giao diện thông tin tài khoản hiện tại của người dùng.
        \item Người dùng chọn chức năng \textit{Chỉnh sửa thông tin}.
        \item Hệ thống hiển thị biểu mẫu (form) cho phép người dùng chỉnh sửa các trường thông tin như: họ tên, email, số điện thoại, địa chỉ, ảnh đại diện, v.v.
        \item Người dùng thay đổi các thông tin mong muốn và nhấn nút \textit{Lưu thay đổi}.
        \item Hệ thống kiểm tra tính hợp lệ của dữ liệu nhập (định dạng email, độ dài, ký tự hợp lệ...).
        \item Nếu dữ liệu hợp lệ, hệ thống cập nhật thông tin mới vào cơ sở dữ liệu.
        \item Hệ thống hiển thị thông báo \textit{Cập nhật thông tin thành công}.
        \item Hệ thống chuyển hướng người dùng về trang quản lý tài khoản với dữ liệu mới.
        \item Kết thúc Use Case.
    \end{enumerate} \\
    \hline
    Luồng sự kiện phụ & 
        \begin{enumerate}[label=\textbf{\arabic*.}, leftmargin=1.2em]
            \item \textbf{Người dùng hủy chỉnh sửa thông tin}
                \begin{enumerate}[label*=\arabic*.]
                    \item Trong quá trình chỉnh sửa, người dùng chọn \textit{Hủy}.
                    \item Hệ thống không lưu bất kỳ thay đổi nào và quay lại trang quản lý tài khoản.
                    \item Kết thúc Use Case.
                \end{enumerate}
            \item \textbf{Dữ liệu nhập không hợp lệ}
                \begin{enumerate}[label*=\arabic*.]
                    \item Người dùng nhập thông tin sai định dạng (ví dụ: email sai, số điện thoại không hợp lệ, để trống trường bắt buộc).
                    \item Hệ thống hiển thị thông báo lỗi cụ thể tại từng trường thông tin.
                    \item Người dùng điều chỉnh lại thông tin và gửi lại yêu cầu.
                \end{enumerate}
            \item \textbf{Email hoặc số điện thoại đã tồn tại}
                \begin{enumerate}[label*=\arabic*.]
                    \item Hệ thống phát hiện email hoặc số điện thoại trùng với tài khoản khác.
                    \item Hệ thống hiển thị thông báo “Email/Số điện thoại đã được sử dụng”.
                    \item Người dùng nhập giá trị khác hoặc hủy thao tác.
                \end{enumerate}
        \end{enumerate} \\
    \hline
    Các yêu cầu đặc biệt & 
    \begin{itemize}
        \item Các trường thông tin (email, số điện thoại) phải được kiểm tra định dạng và tính duy nhất trước khi lưu.
        \item Ảnh đại diện nếu có, phải được kiểm tra định dạng tệp và dung lượng tối đa.
        \item Hệ thống ghi log mọi thay đổi thông tin để phục vụ theo dõi và bảo mật.
        \item Các thay đổi quan trọng (email, số điện thoại) có thể yêu cầu xác minh qua mã OTP hoặc email xác nhận.
    \end{itemize} \\
    \hline
\end{longtable}


\subsection{Usecase tìm kiếm đối tượng}
\begin{figure}[H]
    \centering
    \includegraphics[scale=0.6]{images/usecase tìm kiếm đối tượng.png}
    \caption{Sơ đồ usecase tìm kiếm đối tượng}
\end{figure}

\begin{longtable}{|m{4cm}|m{10cm}|}
    \hline
    \textbf{Tên Use Case} & \textbf{tìm kiếm đối tượng} \\
    \hline
    Tác nhân & Người dùng \\
    \hline
    Mô tả & Tác nhân sử dụng usecase này để tìm kiếm thông tin trong hệ thống\\
    \hline
    Điều kiện kích hoạt & Chọn chức năng tìm kiếm ở hệ thống \\
    \hline
    Tiền điệu kiện & Tác nhân có quyền xem đối tượng \\
    \hline
    Hậu diều kiện & 
    \begin{itemize}
        \item Nếu thành công: Danh sách kết quả được hiển thị; khi chọn một mục sẽ chuyển hướng đến trang chi tiết hoặc kết quả tìm kiếm
        \item Nếu thất bại: không thay đổi dữ liệu. Hệ thống sẽ thông báo lỗi phù hợp
    \end{itemize} \\
    \hline
    Luồng sự kiện chính & 
    \begin{enumerate}[label=\arabic{*}.]
        \item Người dùng chọn vào ô tìm kiếm.
        \item Người dùng nhập thông tin tìm kiếm.
        \item Hệ thống hiển thị danh sách kết quả.
        \item Người dùng chọn một kết quả.
        \item Hệ thống chuyển hướng đến trang kết quả hoặc chi tiết.
        \item Kết thúc usecase.
    \end{enumerate} \\
    \hline
    Luồng sự kiện phụ & 
        \begin{enumerate}[label=\textbf{\arabic*.}, leftmargin=1.2em]
        \item \textbf{Không có kết quả}
            \begin{enumerate}[label*=\arabic*.]
                \item Hệ thống trả về danh sách rỗng.
                \item Hệ thống hiển thị thông tin danh sách tìm kiếm trống.
            \end{enumerate}
        \end{enumerate} \\
    \hline
    Các yêu cầu đặc biệt & Không có \\
    \hline
\end{longtable}


\subsection{Usecase xem chi tiết sản phẩm}
\begin{figure}[H]
    \centering
    \includegraphics[scale=0.6]{images/usecase xem chi tiết sản phẩm.png}
    \caption{Sơ đồ usecase xem chi tiết sản phẩm}
\end{figure}

\begin{longtable}{|m{4cm}|m{10cm}|}
    \hline
    \textbf{Tên Use Case} & \textbf{xem chi tiết sản phẩm} \\
    \hline
    Tác nhân & Khách hàng \\
    \hline
    Mô tả & Tác nhân sử dụng usecase này để xem chi tiết về thông tin của một sản phẩm cụ thể\\
    \hline
    Điều kiện kích hoạt & Tác nhân chọn một sản phẩm cụ thể để xem chi tiết  \\
    \hline
    Tiền điệu kiện & Tác nhân có quyền xem thông tin của sản phẩm \\
    \hline
    Hậu diều kiện & 
    \begin{itemize}
        \item Nếu thành công: Danh sách kết quả được hiển thị; khi chọn một mục sẽ chuyển hướng đến trang chi tiết hoặc kết quả tìm kiếm
        \item Nếu thất bại: không thay đổi dữ liệu. Hệ thống sẽ thông báo lỗi phù hợp
    \end{itemize} \\
    \hline
    Luồng sự kiện chính & 
    \begin{enumerate}[label=\arabic{*}.]
        \item Người dùng chọn vào nút xem chi tiết ở sản phẩm.
        \item Hệ thống hiển thị trang chi tiết sản phẩm.
        \item Kết thúc usecase.
    \end{enumerate} \\
    \hline
    Luồng sự kiện phụ & Không có \\
    \hline
    Các yêu cầu đặc biệt & Không có \\
    \hline
\end{longtable}

\subsection{Usecase Quản lý giỏ hàng}
\begin{figure}[H]
    \centering
    \includegraphics[scale=0.6]{images/usecase quản lí giỏ hàng.png}
    \caption{Sơ đồ Use Case quản lý giỏ hàng}
\end{figure}

\begin{longtable}{|m{4cm}|m{10cm}|}
    \hline
    \textbf{Tên Use Case} & \textbf{Quản lý giỏ hàng} \\
    \hline
    Tác nhân & Khách hàng \\
    \hline
    Mô tả & 
    Cho phép khách hàng thêm sản phẩm vào giỏ hàng, chỉnh sửa số lượng, xóa sản phẩm, và xem tổng giá trị đơn hàng trước khi tiến hành thanh toán. \\
    \hline
    Điều kiện kích hoạt & 
    Khách hàng truy cập vào giỏ hàng hoặc thực hiện hành động thêm sản phẩm vào giỏ hàng từ trang chi tiết sản phẩm. \\
    \hline
    Tiền điều kiện &
    \begin{itemize}
        \item Khách hàng đã đăng nhập.
        \item Hệ thống có dữ liệu sản phẩm hợp lệ.
    \end{itemize} \\
    \hline
    Hậu điều kiện &
    \begin{itemize}
        \item Nếu thành công: Giỏ hàng được cập nhật (thêm/xóa/sửa số lượng), hiển thị tổng tiền chính xác.
        \item Nếu thất bại: Giỏ hàng không thay đổi, hệ thống hiển thị thông báo lỗi phù hợp.
    \end{itemize} \\
    \hline
    Luồng sự kiện chính &
    \begin{enumerate}[label=\arabic*.]
        \item Khách hàng truy cập trang giỏ hàng hoặc chọn “Thêm vào giỏ hàng” từ trang chi tiết sản phẩm.
        \item Hệ thống kiểm tra sản phẩm có tồn tại và còn hàng hay không.
        \item Nếu hợp lệ, hệ thống thêm sản phẩm vào giỏ hàng hoặc cập nhật số lượng (nếu đã có).
        \item Khách hàng có thể:
        \begin{itemize}
            \item Chỉnh sửa số lượng từng sản phẩm.
            \item Xóa sản phẩm khỏi giỏ hàng.
            \item Xem tổng giá trị giỏ hàng (bao gồm thuế, giảm giá nếu có).
        \end{itemize}
        \item Hệ thống hiển thị giỏ hàng đã cập nhật và tổng tiền tạm tính.
        \item Khách hàng chọn “Thanh toán” để chuyển sang quy trình đặt hàng.
        \item Kết thúc Use Case.
    \end{enumerate} \\
    \hline
    Luồng sự kiện phụ &
    \begin{enumerate}[label=\textbf{\arabic*.}, leftmargin=1.2em]
        \item \textbf{Sản phẩm hết hàng}
            \begin{enumerate}[label*=\arabic*.]
                \item Khách hàng thêm sản phẩm nhưng hệ thống phát hiện đã hết hàng.
                \item Hệ thống hiển thị thông báo “Sản phẩm đã hết hàng” và không thêm vào giỏ.
            \end{enumerate}
        \item \textbf{Cập nhật số lượng không hợp lệ}
            \begin{enumerate}[label*=\arabic*.]
                \item Khách hàng nhập số lượng vượt quá tồn kho hoặc nhỏ hơn 1.
                \item Hệ thống hiển thị thông báo “Số lượng không hợp lệ” và giữ giá trị cũ.
            \end{enumerate}
        \item \textbf{Giỏ hàng trống}
            \begin{enumerate}[label*=\arabic*.]
                \item Khách hàng mở giỏ hàng nhưng chưa thêm sản phẩm nào.
                \item Hệ thống hiển thị thông báo “Giỏ hàng của bạn đang trống” và gợi ý xem sản phẩm.
            \end{enumerate}
        \item \textbf{Lỗi hệ thống hoặc mất kết nối}
            \begin{enumerate}[label*=\arabic*.]
                \item Khi thêm/xóa sản phẩm, kết nối đến máy chủ thất bại.
                \item Hệ thống hiển thị thông báo lỗi “Không thể cập nhật giỏ hàng, vui lòng thử lại sau”.
            \end{enumerate}
    \end{enumerate} \\
    \hline
    Các yêu cầu đặc biệt &
    \begin{itemize}
        \item Giỏ hàng cần được lưu tự động trong tài khoản khách hàng.
        \item Cập nhật tổng giá trị giỏ hàng theo thời gian thực khi thay đổi số lượng.
        \item Cho phép tiếp tục mua hàng hoặc chuyển đến thanh toán dễ dàng.
        \item Kiểm tra tồn kho và giá sản phẩm mỗi khi khách hàng mở lại giỏ hàng.
    \end{itemize} \\
    \hline
\end{longtable}



\subsection{Usecase Quản lý danh sách yêu thích}
\begin{figure}[H]
    \centering
    \includegraphics[scale=0.6]{images/usecase quản lí danh sách yêu thích.png}
    \caption{Sơ đồ Use Case quản lý danh sách yêu thích (dành cho khách hàng)}
\end{figure}

\begin{longtable}{|m{4cm}|m{10cm}|}
    \hline
    \textbf{Tên Use Case} & \textbf{Quản lý danh sách yêu thích} \\
    \hline
    Tác nhân & Khách hàng \\
    \hline
    Mô tả & 
    Cho phép khách hàng thêm hoặc xóa sản phẩm khỏi danh sách yêu thích, xem danh sách yêu thích của mình và chuyển sản phẩm từ yêu thích sang giỏ hàng. \\
    \hline
    Điều kiện kích hoạt &
    Khách hàng chọn hành động ``Thêm vào yêu thích'' trên trang chi tiết sản phẩm hoặc mở trang ``Danh sách yêu thích''. \\
    \hline
    Tiền điều kiện &
    \begin{itemize}
        \item Khách hàng đã đăng nhập vào hệ thống.
        \item Sản phẩm còn tồn tại và có thể yêu thích.
    \end{itemize} \\
    \hline
    Hậu điều kiện &
    \begin{itemize}
        \item Nếu thành công: Danh sách yêu thích được cập nhật và hiển thị chính xác trên giao diện của khách hàng.
        \item Nếu thất bại: Danh sách không thay đổi, hệ thống hiển thị thông báo lỗi.
    \end{itemize} \\
    \hline
    Luồng sự kiện chính &
    \begin{enumerate}[label=\arabic*.]
        \item Khách hàng đăng nhập và mở trang chi tiết sản phẩm.
        \item Khách hàng nhấn nút \textit{Thêm vào yêu thích}.
        \item Hệ thống kiểm tra trạng thái sản phẩm (đã có trong yêu thích hay chưa).
        \item Nếu chưa có, hệ thống thêm sản phẩm vào danh sách yêu thích của khách hàng.
        \item Hệ thống cập nhật giao diện.
        \item Khách hàng có thể mở trang danh sách yêu thích để xem các sản phẩm đã lưu.
        \item Tại đây, khách hàng có thể xóa sản phẩm khỏi yêu thích hoặc chuyển sang giỏ hàng.
        \item Hệ thống xử lý hành động tương ứng và hiển thị danh sách cập nhật.
        \item Kết thúc Use Case.
    \end{enumerate} \\
    \hline
    Luồng sự kiện phụ &
    \begin{enumerate}[label=\textbf{\arabic*.}, leftmargin=1.2em]
        \item \textbf{Khách hàng chưa đăng nhập}
            \begin{enumerate}[label*=\arabic*.]
                \item Khách hàng nhấn ``Thêm vào yêu thích'' khi chưa đăng nhập.
                \item Hệ thống chuyển hướng đến trang đăng nhập và yêu cầu đăng nhập trước khi thao tác.
                \item Kết thúc usecase
            \end{enumerate}
        \item \textbf{Sản phẩm đã có trong danh sách yêu thích}
            \begin{enumerate}[label*=\arabic*.]
                \item Hệ thống phát hiện sản phẩm đã nằm trong danh sách.
                \item Hệ thống cho phép bỏ yêu thích (xóa sản phẩm khỏi danh sách).
                \item Kết thúc usecase
            \end{enumerate}
        \item \textbf{Sản phẩm không còn khả dụng}
            \begin{enumerate}[label*=\arabic*.]
                \item Sản phẩm trong danh sách bị ngừng kinh doanh hoặc hết hàng.
                \item Hệ thống hiển thị thông báo và gợi ý xóa khỏi danh sách.
                \item Kết thúc usecase
            \end{enumerate}
    \end{enumerate} \\
    \hline
    Các yêu cầu đặc biệt &
    \begin{itemize}
        \item Danh sách yêu thích được lưu riêng cho từng tài khoản khách hàng.
        \item Trạng thái yêu thích được cập nhật theo thời gian thực.
        \item Hệ thống kiểm tra tồn kho khi hiển thị danh sách yêu thích.
        \item Hệ thống chỉ cho phép khách hàng đã đăng nhập thực hiện thao tác yêu thích.
    \end{itemize} \\
    \hline
\end{longtable}


\subsection{Usecase Đặt hàng}
\begin{figure}[H]
    \centering
    \includegraphics[scale=0.6]{images/usecase đặt hàng.png}
    \caption{Sơ đồ Use Case Đặt hàng}
\end{figure}

\begin{longtable}{|m{4cm}|m{10cm}|}
    \hline
    \textbf{Tên Use Case} & \textbf{Đặt hàng} \\
    \hline
    Tác nhân & Khách hàng \\
    \hline
    Mô tả &
    Cho phép khách hàng xác nhận giỏ hàng, chọn địa chỉ giao hàng, chọn phương thức vận chuyển, (tuỳ chọn) áp dụng mã giảm giá, xem tóm tắt đơn và thực hiện thanh toán để hoàn tất đơn hàng. \\
    \hline
    Điều kiện kích hoạt &
    Khách hàng chọn hành động ``Đặt hàng/Checkout'' từ trang giỏ hàng. \\
    \hline
    Tiền điều kiện &
    \begin{itemize}
        \item Khách hàng đã đăng nhập.
        \item Giỏ hàng có ít nhất một sản phẩm hợp lệ.
        \item Hệ thống sẵn sàng (kết nối CSDL, dịch vụ thanh toán/ship còn hoạt động).
    \end{itemize} \\
    \hline
    Hậu điều kiện &
    \begin{itemize}
        \item Nếu thành công: Đơn hàng được tạo với trạng thái ban đầu (ví dụ: \textit{Chờ thanh toán}/\textit{Đã thanh toán} tùy phương thức); tồn kho được trừ/giữ tạm; gửi thông báo xác nhận.
        \item Nếu thất bại: Không tạo đơn hoặc huỷ đơn tạm; giỏ hàng giữ nguyên; hiển thị thông báo lỗi phù hợp.
    \end{itemize} \\
    \hline
    Luồng sự kiện chính &
    \begin{enumerate}[label=\arabic*.]
        \item Hệ thống hiển thị trang \textit{Xác nhận giỏ hàng} (danh sách sản phẩm, số lượng, giá, phí tạm tính).
        \item Khách hàng chọn \textit{Địa chỉ giao hàng} từ danh sách địa chỉ có sẵn.
        \item Hệ thống \textbf{<<include>>} kiểm tra tồn kho và tính lại tổng tiền (nếu có thay đổi giá/phí).
        \item Hệ thống \textbf{<<include>>} hiển thị và cho phép chọn \textit{Phương thức vận chuyển} (phí, thời gian dự kiến).
        \item (Tuỳ chọn) Khách hàng nhập \textit{Mã giảm giá}; hệ thống \textbf{<<extend>>} \textit{Áp dụng mã giảm giá}.
        \item Hệ thống hiển thị \textit{Tóm tắt đơn hàng} (sản phẩm, địa chỉ, vận chuyển, giảm giá, tổng thanh toán).
        \item Khách hàng xác nhận ``Đặt hàng''.
        \item Hệ thống tạo \textit{đơn hàng tạm} và \textbf{giữ tạm tồn kho} trong thời gian cho phép.
        \item Hệ thống \textbf{<<include>>} chuyển sang \textit{Thanh toán} theo phương thức đã chọn.
        \item Sau khi thanh toán thành công, hệ thống cập nhật trạng thái đơn, gửi xác nhận, và hiển thị trang \textit{Đặt hàng thành công}.
        \item Kết thúc Use Case.
    \end{enumerate} \\
    \hline
    Luồng sự kiện phụ &
    \begin{enumerate}[label=\textbf{\arabic*.}, leftmargin=1.2em]
        \item \textbf{Chưa có địa chỉ phù hợp}
            \begin{enumerate}[label*=\arabic*.]
                \item Khách hàng không có địa chỉ hoặc muốn sửa hoặc thêm địa chỉ mới.
                \item Hệ thống chọn chức năng phù hợp trong quản lí địa chỉ giao hàng để thêm hoặc sửa địa chỉ.
                \item Quay lại checkout và tiếp tục.
            \end{enumerate}
        \item \textbf{Áp dụng mã giảm giá thất bại}
            \begin{enumerate}[label*=\arabic*.]
                \item Mã hết hạn/không hợp lệ hoặc không đáp ứng điều kiện.
                \item Hệ thống thông báo lỗi, không áp dụng giảm giá; khách hàng có thể nhập mã khác hoặc bỏ qua.
            \end{enumerate}
        \item \textbf{Hết hàng hoặc thay đổi giá trong lúc đặt hàng}
            \begin{enumerate}[label*=\arabic*.]
                \item Hệ thống phát hiện một số sản phẩm hết tồn hoặc giá hoặc khuyến mãi thay đổi.
                \item Hệ thống thông báo và yêu cầu khách hàng xác nhận lại; cho phép quay về giỏ hàng để điều chỉnh.
            \end{enumerate}
        \item \textbf{Thanh toán thất bại hoặc hủy giữa chừng}
            \begin{enumerate}[label*=\arabic*.]
                \item Giao dịch bị từ chối hoặc hết thời gian hoặc khách hủy.
                \item Hệ thống cập nhật đơn về trạng thái phù hợp; giải phóng tồn kho giữ tạm.
            \end{enumerate}
    \end{enumerate} \\
    \hline
    Các yêu cầu đặc biệt &
    \begin{itemize}
        \item \textbf{Bắt buộc} chọn phương thức vận chuyển và thực hiện thanh toán trong luồng đặt hàng và thanh toán.
        \item \textbf{Tùy chọn} quản lý địa chỉ, áp mã giảm giá trong đặt hàng.
        \item Giữ tạm tồn kho trong thời gian giới hạn; tự động giải phóng nếu thanh toán thất bại/hết hạn.
        \item Tính toán tổng tiền theo thời gian thực (phí ship, thuế, giảm giá); chống đua giá bằng cách đóng dấu phiên bản báo giá.
    \end{itemize} \\
    \hline
\end{longtable}


\subsection{Usecase Theo dõi đơn hàng}
\begin{figure}[H]
    \centering
    \includegraphics[scale=0.6]{images/usecase theo dõi đơn hàng.png}
    \caption{Sơ đồ Use Case Theo dõi đơn hàng}
\end{figure}

\begin{longtable}{|m{4cm}|m{10cm}|}
    \hline
    \textbf{Tên Use Case} & \textbf{Theo dõi đơn hàng} \\
    \hline
    Tác nhân & Khách hàng \\
    \hline
    Mô tả &
    Cho phép khách hàng xem danh sách đơn hàng của mình, trạng thái vận chuyển theo thời gian thực, lọc/tìm kiếm đơn; từ đây có thể thực hiện các hành động mở rộng: \textit{Xem chi tiết}, \textit{Hủy đơn hàng} (khi còn cho phép), \textit{Xác nhận Đã nhận hàng} (khi đã giao). \\
    \hline
    Điều kiện kích hoạt &
    Khách hàng truy cập mục \textit{Theo dõi đơn hàng} trong tài khoản hoặc từ thông báo/liên kết lịch sử mua hàng. \\
    \hline
    Tiền điều kiện &
    \begin{itemize}
        \item Khách hàng đã đăng nhập.
        \item Hệ thống có dữ liệu đơn hàng của khách (ít nhất là 0 đơn).
    \end{itemize} \\
    \hline
    Hậu điều kiện &
    \begin{itemize}
        \item Nếu thành công: Danh sách đơn hiển thị; trạng thái mỗi đơn được cập nhật; các thao tác mở rộng được kích hoạt theo điều kiện trạng thái.
        \item Nếu thất bại: Không thay đổi dữ liệu; hiển thị thông báo lỗi phù hợp.
    \end{itemize} \\
    \hline
    Luồng sự kiện chính &
    \begin{enumerate}[label=\arabic*.]
        \item Hệ thống hiển thị danh sách đơn hàng của khách: mã đơn, ngày tạo, tổng tiền, trạng thái hiện tại, tiến trình vận chuyển.
        \item Khách hàng có thể tìm kiếm/lọc theo mã đơn, trạng thái, thời gian.
        \item Khách hàng chọn một đơn để \textbf{Xem chi tiết}.
        \item Nếu trạng thái đơn cho phép hủy (ví dụ: \textit{Chờ xử lý}), khách có thể chọn \textbf{Hủy đơn hàng}.
        \item Nếu trạng thái đơn là \textit{Đã giao}, khách có thể chọn \textbf{Đã nhận hàng}.
        \item Hệ thống cập nhật giao diện sau mỗi hành động; tiếp tục cho phép theo dõi các đơn khác.
        \item Kết thúc Use Case.
    \end{enumerate} \\
    \hline
    Luồng sự kiện phụ &
    \begin{enumerate}[label=\textbf{\arabic*.}, leftmargin=1.2em]
        \item \textbf{Không có đơn hàng}
            \begin{enumerate}[label*=\arabic*.]
                \item Hệ thống hiển thị danh sách rỗng và gợi ý quay lại mua sắm.
            \end{enumerate}
        \item \textbf{Đơn không thuộc về khách hàng}
            \begin{enumerate}[label*=\arabic*.]
                \item Khách mở liên kết đơn không thuộc tài khoản.
                \item Hệ thống từ chối truy cập và hiển thị thông báo quyền hạn.
            \end{enumerate}
        \item \textbf{Dữ liệu trạng thái trễ hoặc dịch vụ vận chuyển lỗi}
            \begin{enumerate}[label*=\arabic*.]
                \item Hệ thống không lấy được cập nhật mới.
                \item Hiển thị trạng thái gần nhất và thông báo thử lại sau.
            \end{enumerate}
    \end{enumerate} \\
    \hline
    Các yêu cầu đặc biệt &
    \begin{itemize}
        \item Đồng bộ trạng thái theo thời gian thực (hoặc gần thực) từ hệ thống xử lý đơn/đơn vị vận chuyển.
        \item Chỉ hiển thị hoặc hỗ trợ hành động phù hợp với trạng thái đơn (RBAC + state guard).
        \item Phân trang hoặc tải thêm; tìm kiếm và lọc hiệu quả trên tập dữ liệu lớn.
    \end{itemize} \\
    \hline
\end{longtable}


\subsection{Usecase Quản lí sản phẩm}
\begin{figure}[H]
    \centering
    \includegraphics[scale=0.6]{images/usecase quản lí sản phẩm.png}
    \caption{Sơ đồ Use Case Quản lí sản phẩm}
\end{figure}

\begin{longtable}{|m{4cm}|m{10cm}|}
    \hline
    \textbf{Tên Use Case} & \textbf{Quản lí sản phẩm} \\
    \hline
    Tác nhân &
    Nhân viên \\
    \hline
    Mô tả &
    Cho phép nhân viên tạo/cập nhật/xoá và tra cứu sản phẩm, quản lí thông tin cơ bản (SKU, tên, mô tả), phân loại, giá, tồn kho, media; quản lí thuộc tính/bộ thuộc tính. \\
    \hline
    Điều kiện kích hoạt &
    Nhân viên chọn chức năng quản lý sản phẩm. \\
    \hline
    Tiền điều kiện & Tác nhân đã đăng nhập và có quyền phù hợp (RBAC). \\
    \hline
    Hậu điều kiện &
    \begin{itemize}
        \item Dữ liệu sản phẩm nhất quán; chỉ mục tìm kiếm được đồng bộ.
        \item Ghi nhận cho các thay đổi.
    \end{itemize} \\
    \hline
    Luồng sự kiện chính &
    \begin{enumerate}[label=\arabic*.]
        \item Tác nhân chọn chức năng quản lý sản phẩm.
        \item Tác nhân chọn một thao tác: \textit{Thêm mới}, \textit{Chỉnh sửa}, \textit{Xoá}, hoặc \textit{Tìm kiếm nâng cao}.
        \item Nếu \textit{Thêm mới}/\textit{Chỉnh sửa}: hiển thị biểu mẫu; tác nhân nhập/điều chỉnh thông tin (SKU, tên, mô tả, danh mục, giá, tồn, media).
        \item Tác nhân chọn lưu thông tin.
        \item Hệ thống Kiểm tra hợp lệ dữ liệu (bắt buộc, định dạng, ràng buộc giá/tồn).
        \item Hệ thống Kiểm tra duy nhất SKU khi tạo mới hoặc đổi SKU.
        \item (Tuỳ chọn) Tác nhân chọn Quản lí thuộc tính và gán cho sản phẩm.
        \item Hệ thống lưu thay đổi;
        \item Kết thúc Use Case.
    \end{enumerate} \\
    \hline
    Luồng sự kiện phụ &
    \begin{enumerate}[label=\textbf{\arabic*.}, leftmargin=1.2em]
        \item \textbf{Dữ liệu không hợp lệ}
            \begin{enumerate}[label*=\arabic*.]
                \item Trường bắt buộc thiếu, giá/tồn âm, định dạng sai.
                \item Hệ thống thông báo lỗi, đánh dấu trường vi phạm và giữ dữ liệu đã nhập.
            \end{enumerate}
        \item \textbf{Hủy bỏ việc quản lý sản phẩm}
            \begin{enumerate}[label*=\arabic*.]
                \item Tác nhân hủy bỏ việc quản lý sản phẩm.
                \item Hệ thống bỏ qua và trở về giao diện chính
                \item 3.	Kết thúc Use Case.
            \end{enumerate}
    \end{enumerate} \\
    \hline
    Các yêu cầu đặc biệt & Không có\\
    \hline
\end{longtable}


\subsection{Usecase Quản lí thuộc tính}
\begin{figure}[H]
    \centering
    \includegraphics[scale=0.6]{images/usecase quản lí thuộc tính.png}
    \caption{Sơ đồ Use Case Quản lí thuộc tính}
\end{figure}

\begin{longtable}{|m{4cm}|m{10cm}|}
    \hline
    \textbf{Tên Use Case} & \textbf{Quản lí thuộc tính} \\
    \hline
    Tác nhân & Nhân viên \\
    \hline
    Mô tả &
    Cho phép nhân viên tạo, chỉnh sửa, xoá và quản lý các thuộc tính của sản phẩm như màu sắc, kích thước, chất liệu... Các thuộc tính này có thể được gán vào từng sản phẩm cụ thể để phục vụ cho việc hiển thị, tìm kiếm và lọc sản phẩm. \\
    \hline
    Điều kiện kích hoạt &
    Nhân viên chọn chức năng \textit{Quản lí thuộc tính} trong hệ thống quản trị. \\
    \hline
    Tiền điều kiện & 
    Tác nhân đã đăng nhập và có quyền truy cập vào chức năng quản lý thuộc tính. \\
    \hline
    Hậu điều kiện &
    \begin{itemize}
        \item Dữ liệu thuộc tính được cập nhật và lưu trữ chính xác trong hệ thống.
        \item Các sản phẩm liên quan được đồng bộ thuộc tính mới (nếu có).
    \end{itemize} \\
    \hline
    Luồng sự kiện chính &
    \begin{enumerate}[label=\arabic*.]
        \item Tác nhân chọn chức năng \textit{Quản lí thuộc tính}.
        \item Hệ thống hiển thị danh sách các thuộc tính hiện có.
        \item Tác nhân chọn thao tác: \textit{Thêm mới}, \textit{Chỉnh sửa} hoặc \textit{Xoá} thuộc tính.
        \item Nếu \textit{Thêm mới}/\textit{Chỉnh sửa}: hệ thống hiển thị biểu mẫu; tác nhân nhập hoặc cập nhật thông tin thuộc tính (tên, kiểu dữ liệu, giá trị mặc định, danh sách lựa chọn...).
        \item Tác nhân chọn \textit{Lưu}.
        \item Hệ thống kiểm tra tính hợp lệ dữ liệu (tên trùng, kiểu không hợp lệ...).
        \item Hệ thống lưu thay đổi và cập nhật danh sách thuộc tính.
        \item Kết thúc Use Case.
    \end{enumerate} \\
    \hline
    Luồng sự kiện phụ &
    \begin{enumerate}[label=\textbf{\arabic*.}, leftmargin=1.2em]
        \item \textbf{Dữ liệu không hợp lệ}
            \begin{enumerate}[label*=\arabic*.]
                \item Tên thuộc tính bị trùng hoặc bỏ trống.
                \item Hệ thống hiển thị thông báo lỗi, yêu cầu nhập lại.
            \end{enumerate}
        \item \textbf{Xóa thuộc tính đang sử dụng}
            \begin{enumerate}[label*=\arabic*.]
                \item Tác nhân chọn xoá thuộc tính đang được gán cho sản phẩm.
                \item Hệ thống hiển thị cảnh báo và yêu cầu xác nhận trước khi thực hiện.
            \end{enumerate}
        \item \textbf{Hủy thao tác}
            \begin{enumerate}[label*=\arabic*.]
                \item Tác nhân hủy bỏ thao tác đang thực hiện.
                \item Hệ thống trở về giao diện danh sách thuộc tính.
            \end{enumerate}
    \end{enumerate} \\
    \hline
    Các yêu cầu đặc biệt & 
    \begin{itemize}
        \item Hỗ trợ các loại thuộc tính cơ bản: văn bản, số, danh sách chọn, ngày giờ.
        \item Tự động kiểm tra trùng tên thuộc tính trước khi lưu.
        \item Cho phép lọc và sắp xếp danh sách thuộc tính theo tên hoặc loại.
    \end{itemize} \\
    \hline
\end{longtable}


\subsection{Usecase Quản lí đơn hàng}
\begin{figure}[H]
    \centering
    \includegraphics[scale=0.6]{images/usecase quản lí đơn hàng.png}
    \caption{Sơ đồ Use Case Quản lí đơn hàng}
\end{figure}

\begin{longtable}{|m{4cm}|m{10cm}|}
    \hline
    \textbf{Tên Use Case} & \textbf{Quản lí đơn hàng} \\
    \hline
    Tác nhân & Nhân viên / Quản trị viên \\
    \hline
    Mô tả &
    Cho phép nhân viên quản lý danh sách đơn hàng, bao gồm: tìm kiếm, xem chi tiết thông tin đơn hàng và duyệt (xác nhận) các đơn chờ xử lý. Hệ thống giúp đảm bảo việc theo dõi và cập nhật trạng thái đơn hàng được chính xác, kịp thời. \\
    \hline
    Điều kiện kích hoạt &
    Nhân viên chọn chức năng \textit{Quản lí đơn hàng} trong giao diện hệ thống. \\
    \hline
    Tiền điều kiện &
    \begin{itemize}
        \item Tác nhân đã đăng nhập và có quyền truy cập quản lý đơn hàng.
        \item Hệ thống kết nối ổn định với cơ sở dữ liệu đơn hàng.
    \end{itemize} \\
    \hline
    Hậu điều kiện &
    \begin{itemize}
        \item Dữ liệu đơn hàng được cập nhật chính xác (sau khi duyệt hoặc thay đổi trạng thái).
        \item Lịch sử thao tác được ghi nhận để truy vết. 
    \end{itemize} \\
    \hline
    Luồng sự kiện chính &
    \begin{enumerate}[label=\arabic*.]
        \item Tác nhân chọn chức năng \textit{Quản lí đơn hàng}.
        \item Hệ thống hiển thị danh sách tất cả các đơn hàng cùng bộ lọc (mã đơn, khách hàng, ngày đặt, trạng thái...).
        \item Tác nhân có thể nhập thông tin tìm kiếm hoặc áp dụng bộ lọc để lọc đơn hàng theo yêu cầu.
        \item Hệ thống hiển thị danh sách kết quả tương ứng.
        \item Tác nhân chọn một đơn hàng cụ thể để xem chi tiết.
        \item Hệ thống hiển thị thông tin chi tiết đơn hàng (sản phẩm, số lượng, giá, khách hàng, địa chỉ giao hàng, trạng thái thanh toán, trạng thái giao hàng...).
        \item Nếu đơn hàng ở trạng thái “Chờ duyệt”, tác nhân chọn thao tác \textit{Duyệt đơn hàng}.
        \item Hệ thống kiểm tra tính hợp lệ (đơn chưa duyệt, thanh toán hợp lệ, hàng còn tồn...).
        \item Hệ thống cập nhật trạng thái đơn sang “Đã duyệt” và ghi nhận thông tin người duyệt.
        \item Kết thúc Use Case.
    \end{enumerate} \\
    \hline
    Luồng sự kiện phụ &
    \begin{enumerate}[label=\textbf{\arabic*.}, leftmargin=1.2em]
        \item \textbf{Không có kết quả tìm kiếm}
            \begin{enumerate}[label*=\arabic*.]
                \item Hệ thống hiển thị thông báo “Không tìm thấy đơn hàng phù hợp”.
                \item Tác nhân có thể thay đổi điều kiện lọc hoặc tìm kiếm khác.
            \end{enumerate}
        \item \textbf{Đơn hàng đã được duyệt trước đó}
            \begin{enumerate}[label*=\arabic*.]
                \item Tác nhân chọn duyệt lại một đơn đã được duyệt.
                \item Hệ thống thông báo đơn hàng này đã được xử lý, không thể duyệt lại.
            \end{enumerate}
        \item \textbf{Lỗi khi cập nhật trạng thái}
            \begin{enumerate}[label*=\arabic*.]
                \item Xảy ra lỗi kết nối hoặc cập nhật cơ sở dữ liệu.
                \item Hệ thống hiển thị thông báo lỗi và giữ nguyên trạng thái cũ.
            \end{enumerate}
        \item \textbf{Hủy thao tác}
            \begin{enumerate}[label*=\arabic*.]
                \item Tác nhân hủy bỏ thao tác duyệt hoặc thoát khỏi màn hình chi tiết đơn hàng.
                \item Hệ thống quay về danh sách đơn hàng ban đầu.
            \end{enumerate}
    \end{enumerate} \\
    \hline
    Các yêu cầu đặc biệt &
    \begin{itemize}
        \item Cho phép tìm kiếm đơn hàng theo nhiều tiêu chí (mã đơn, tên khách hàng, ngày, trạng thái...).
        \item Hỗ trợ phân trang, sắp xếp và lọc theo trạng thái đơn hàng.
        \item Khi duyệt đơn, hệ thống tự động ghi lại thời gian duyệt và người thực hiện.
        \item Bảo đảm tính toàn vẹn dữ liệu khi cập nhật trạng thái đơn hàng.
    \end{itemize} \\
    \hline
\end{longtable}


\subsection{Usecase Quản lí danh mục}
\begin{figure}[H]
    \centering
    \includegraphics[scale=0.6]{images/usecase quản lí danh mục.png}
    \caption{Sơ đồ Use Case Quản lí danh mục}
\end{figure}

\begin{longtable}{|m{4cm}|m{10cm}|}
    \hline
    \textbf{Tên Use Case} & \textbf{Quản lí danh mục} \\
    \hline
    Tác nhân & Nhân viên \\
    \hline
    Mô tả &
    Cho phép nhân viên tạo, chỉnh sửa, xoá và tra cứu danh mục sản phẩm. Danh mục giúp tổ chức sản phẩm theo nhóm (ví dụ: Áo, Quần, Giày, Phụ kiện) để hỗ trợ việc hiển thị, tìm kiếm và lọc trong hệ thống. \\
    \hline
    Điều kiện kích hoạt &
    Nhân viên chọn chức năng \textit{Quản lí danh mục} trong giao diện quản trị. \\
    \hline
    Tiền điều kiện &
    \begin{itemize}
        \item Tác nhân đã đăng nhập và có quyền truy cập quản lý danh mục.
        \item Hệ thống kết nối ổn định với cơ sở dữ liệu.
    \end{itemize} \\
    \hline
    Hậu điều kiện &
    \begin{itemize}
        \item \textbf{Nếu thành công:}  
        Danh mục được tạo, chỉnh sửa hoặc xoá thành công; dữ liệu được cập nhật chính xác trong hệ thống; các sản phẩm liên quan được đồng bộ lại với danh mục mới (nếu có thay đổi); hiển thị thông báo thao tác thành công.
        \item \textbf{Nếu thất bại:}  
        Thông tin danh mục không được thay đổi; hệ thống hiển thị thông báo lỗi (ví dụ: tên trùng, danh mục cha không hợp lệ, lỗi kết nối...); người dùng có thể sửa lại hoặc hủy thao tác.
    \end{itemize} \\
    \hline
    Luồng sự kiện chính &
    \begin{enumerate}[label=\arabic*.]
        \item Tác nhân chọn chức năng \textit{Quản lí danh mục}.
        \item Hệ thống hiển thị danh sách danh mục hiện có theo dạng cây (nếu có phân cấp).
        \item Tác nhân chọn thao tác: \textit{Thêm mới}, \textit{Chỉnh sửa}, hoặc \textit{Xoá} danh mục.
        \item Nếu \textit{Thêm mới}/\textit{Chỉnh sửa}: hệ thống hiển thị biểu mẫu; tác nhân nhập thông tin danh mục (tên, mô tả, danh mục cha, hình ảnh đại diện...).
        \item Tác nhân chọn \textit{Lưu}.
        \item Hệ thống kiểm tra tính hợp lệ (tên trùng, danh mục cha không hợp lệ...).
        \item Hệ thống lưu thay đổi và cập nhật danh sách danh mục.
        \item Kết thúc Use Case.
    \end{enumerate} \\
    \hline
    Luồng sự kiện phụ &
    \begin{enumerate}[label=\textbf{\arabic*.}, leftmargin=1.2em]
        \item \textbf{Tên danh mục bị trùng}
            \begin{enumerate}[label*=\arabic*.]
                \item Tác nhân nhập tên danh mục đã tồn tại.
                \item Hệ thống hiển thị thông báo lỗi và yêu cầu nhập tên khác.
            \end{enumerate}
        \item \textbf{Danh mục cha không hợp lệ}
            \begin{enumerate}[label*=\arabic*.]
                \item Tác nhân chọn danh mục cha là chính danh mục hiện tại hoặc cấp con của nó.
                \item Hệ thống hiển thị cảnh báo và không cho lưu.
            \end{enumerate}
        \item \textbf{Xoá danh mục có sản phẩm con}
            \begin{enumerate}[label*=\arabic*.]
                \item Tác nhân chọn xoá danh mục đang chứa sản phẩm.
                \item Hệ thống cảnh báo và yêu cầu chuyển sản phẩm sang danh mục khác trước khi xoá.
            \end{enumerate}
        \item \textbf{Hủy thao tác}
            \begin{enumerate}[label*=\arabic*.]
                \item Tác nhân hủy bỏ thao tác đang thực hiện.
                \item Hệ thống quay lại danh sách danh mục ban đầu.
            \end{enumerate}
    \end{enumerate} \\
    \hline
    Các yêu cầu đặc biệt &
    \begin{itemize}
        \item Cho phép quản lý danh mục theo cấu trúc phân cấp (cha – con).
        \item Hỗ trợ tải hình ảnh biểu tượng cho danh mục.
        \item Kiểm tra trùng tên danh mục theo từng cấp.
        \item Khi thay đổi danh mục cha, hệ thống tự động cập nhật cấu trúc liên quan.
    \end{itemize} \\
    \hline
\end{longtable}


\subsection{Usecase Quản lí khuyến mãi}
\begin{figure}[H]
    \centering
    \includegraphics[scale=0.6]{images/usecase quản lí khuyến mãi.png}
    \caption{Sơ đồ Use Case Quản lí khuyến mãi}
\end{figure}

\begin{longtable}{|m{4cm}|m{10cm}|}
    \hline
    \textbf{Tên Use Case} & \textbf{Quản lí khuyến mãi} \\
    \hline
    Tác nhân & Nhân viên \\
    \hline
    Mô tả &
    Cho phép nhân viên tạo, chỉnh sửa, xoá và quản lý các chương trình khuyến mãi của hệ thống. Khuyến mãi có thể áp dụng cho toàn bộ sản phẩm, một nhóm sản phẩm hoặc một danh mục cụ thể, nhằm thúc đẩy doanh số và thu hút khách hàng. \\
    \hline
    Điều kiện kích hoạt &
    Nhân viên chọn chức năng \textit{Quản lí khuyến mãi} trong giao diện quản trị hệ thống. \\
    \hline
    Tiền điều kiện &
    \begin{itemize}
        \item Tác nhân đã đăng nhập và có quyền truy cập vào chức năng quản lý khuyến mãi.
        \item Danh sách sản phẩm hoặc danh mục áp dụng khuyến mãi đã tồn tại trong hệ thống.
    \end{itemize} \\
    \hline
    Hậu điều kiện &
    \begin{itemize}
        \item \textbf{Nếu thành công:}  
        Chương trình khuyến mãi được lưu hoặc cập nhật thành công; thông tin được đồng bộ tới các sản phẩm liên quan; hiển thị đúng trên giao diện bán hàng; ghi nhận lịch sử thao tác.
        \item \textbf{Nếu thất bại:}  
        Dữ liệu không được thay đổi; hiển thị thông báo lỗi phù hợp (ví dụ: trùng tên, sai thời gian, điều kiện không hợp lệ); người dùng có thể nhập lại hoặc huỷ thao tác.
    \end{itemize} \\
    \hline
    Luồng sự kiện chính &
    \begin{enumerate}[label=\arabic*.]
        \item Tác nhân chọn chức năng \textit{Quản lí khuyến mãi}.
        \item Hệ thống hiển thị danh sách các chương trình khuyến mãi hiện có.
        \item Tác nhân chọn thao tác: \textit{Thêm mới}, \textit{Chỉnh sửa}, hoặc \textit{Xoá} khuyến mãi.
        \item Nếu \textit{Thêm mới}/\textit{Chỉnh sửa}: hệ thống hiển thị biểu mẫu; tác nhân nhập thông tin khuyến mãi gồm:
        \begin{itemize}
            \item Tên chương trình.
            \item Mô tả.
            \item Loại khuyến mãi (giảm theo %, giá cố định, mua X tặng Y, miễn phí vận chuyển...).
            \item Sản phẩm/danh mục áp dụng.
            \item Ngày bắt đầu, ngày kết thúc.
            \item Điều kiện áp dụng (đơn tối thiểu, số lượng tối đa, mã giảm giá...).
        \end{itemize}
        \item Tác nhân chọn \textit{Lưu}.
        \item Hệ thống kiểm tra tính hợp lệ (tên trùng, thời gian, điều kiện áp dụng...).
        \item Nếu hợp lệ, hệ thống lưu thông tin và cập nhật danh sách khuyến mãi.
        \item Kết thúc Use Case.
    \end{enumerate} \\
    \hline
    Luồng sự kiện phụ &
    \begin{enumerate}[label=\textbf{\arabic*.}, leftmargin=1.2em]
        \item \textbf{Thông tin không hợp lệ}
            \begin{enumerate}[label*=\arabic*.]
                \item Tên chương trình bị trùng, thời gian kết thúc sớm hơn thời gian bắt đầu hoặc phần trăm giảm vượt giới hạn.
                \item Hệ thống hiển thị thông báo lỗi và yêu cầu chỉnh sửa.
            \end{enumerate}
        \item \textbf{Xoá khuyến mãi đang hoạt động}
            \begin{enumerate}[label*=\arabic*.]
                \item Tác nhân chọn xoá chương trình đang còn hiệu lực.
                \item Hệ thống cảnh báo và yêu cầu xác nhận trước khi vô hiệu hoá chương trình.
            \end{enumerate}
        \item \textbf{Khuyến mãi tự động hết hạn}
            \begin{enumerate}[label*=\arabic*.]
                \item Khi đến ngày kết thúc, hệ thống tự động chuyển trạng thái sang “Hết hạn”.
                \item Các sản phẩm liên quan trở lại giá gốc.
            \end{enumerate}
        \item \textbf{Hủy thao tác}
            \begin{enumerate}[label*=\arabic*.]
                \item Tác nhân huỷ bỏ thao tác thêm hoặc sửa.
                \item Hệ thống quay lại danh sách khuyến mãi mà không lưu thay đổi.
            \end{enumerate}
    \end{enumerate} \\
    \hline
    Các yêu cầu đặc biệt &
    \begin{itemize}
        \item Cho phép lọc, tìm kiếm khuyến mãi theo tên, trạng thái, thời gian hiệu lực.
        \item Hỗ trợ nhiều loại khuyến mãi (theo sản phẩm, danh mục, hóa đơn, mã giảm giá).
        \item Tự động cập nhật trạng thái khi hết hạn hoặc bị vô hiệu hóa.
        \item Ngăn trùng lặp các chương trình cùng loại trên cùng sản phẩm.
        \item Ghi nhận lịch sử thay đổi và người thực hiện thao tác.
    \end{itemize} \\
    \hline
\end{longtable}


\subsection{Usecase Quản lí phân quyền}
\begin{figure}[H]
    \centering
    \includegraphics[scale=0.6]{images/usecase quản lí phân quyền.png}
    \caption{Sơ đồ Use Case Quản lí phân quyền}
\end{figure}

\begin{longtable}{|m{4cm}|m{10cm}|}
    \hline
    \textbf{Tên Use Case} & \textbf{Quản lí phân quyền} \\
    \hline
    Tác nhân & Quản trị viên \\
    \hline
    Mô tả &
    Cho phép quản trị viên tạo, chỉnh sửa, xoá và gán quyền truy cập cho người dùng trong hệ thống. Chức năng này giúp đảm bảo mỗi người dùng chỉ được phép thực hiện các thao tác phù hợp với vai trò được phân công, đảm bảo tính bảo mật và toàn vẹn dữ liệu. \\
    \hline
    Điều kiện kích hoạt &
    Quản trị viên chọn chức năng \textit{Quản lí phân quyền} trong giao diện quản trị hệ thống. \\
    \hline
    Tiền điều kiện &
    \begin{itemize}
        \item Quản trị viên đã đăng nhập vào hệ thống.
        \item Có quyền truy cập vào chức năng phân quyền.
        \item Danh sách người dùng đã được tạo trong hệ thống.
    \end{itemize} \\
    \hline
    Hậu điều kiện &
    \begin{itemize}
        \item \textbf{Nếu thành công:}  
        Thông tin vai trò và quyền hạn của người dùng được cập nhật chính xác trong hệ thống; quyền truy cập mới có hiệu lực ngay; hệ thống ghi nhận nhật ký thao tác của quản trị viên.
        \item \textbf{Nếu thất bại:}  
        Dữ liệu không thay đổi; hệ thống hiển thị thông báo lỗi (ví dụ: vai trò trùng tên, quyền không hợp lệ, lỗi kết nối...); quản trị viên có thể chỉnh sửa lại hoặc huỷ thao tác.
    \end{itemize} \\
    \hline
    Luồng sự kiện chính &
    \begin{enumerate}[label=\arabic*.]
        \item Quản trị viên chọn chức năng \textit{Quản lí phân quyền}.
        \item Hệ thống hiển thị danh sách các vai trò hiện có và danh sách người dùng trong hệ thống.
        \item Quản trị viên chọn thao tác: \textit{Thêm mới vai trò}, \textit{Chỉnh sửa quyền}, hoặc \textit{Xoá vai trò}.
        \item Nếu \textit{Thêm mới}/\textit{Chỉnh sửa}: hệ thống hiển thị biểu mẫu cho phép nhập thông tin vai trò (tên, mô tả) và chọn các quyền tương ứng (xem, thêm, sửa, xoá...).
        \item Quản trị viên chọn \textit{Lưu}.
        \item Hệ thống kiểm tra tính hợp lệ (tên vai trò trùng, quyền không hợp lệ...).
        \item Hệ thống lưu thay đổi và cập nhật quyền truy cập cho người dùng có vai trò đó.
        \item Kết thúc Use Case.
    \end{enumerate} \\
    \hline
    Luồng sự kiện phụ &
    \begin{enumerate}[label=\textbf{\arabic*.}, leftmargin=1.2em]
        \item \textbf{Tên vai trò bị trùng}
            \begin{enumerate}[label*=\arabic*.]
                \item Quản trị viên nhập tên vai trò đã tồn tại trong hệ thống.
                \item Hệ thống hiển thị thông báo lỗi và yêu cầu nhập tên khác.
            \end{enumerate}
        \item \textbf{Quyền không hợp lệ}
            \begin{enumerate}[label*=\arabic*.]
                \item Quản trị viên chọn quyền không có trong danh sách hoặc vượt quyền cho phép.
                \item Hệ thống hiển thị cảnh báo và không cho lưu.
            \end{enumerate}
        \item \textbf{Xoá vai trò đang được sử dụng}
            \begin{enumerate}[label*=\arabic*.]
                \item Quản trị viên chọn xoá vai trò đang được gán cho người dùng.
                \item Hệ thống cảnh báo và yêu cầu chuyển người dùng sang vai trò khác trước khi xoá.
            \end{enumerate}
        \item \textbf{Hủy thao tác}
            \begin{enumerate}[label*=\arabic*.]
                \item Quản trị viên huỷ thao tác thêm hoặc sửa vai trò.
                \item Hệ thống quay lại danh sách vai trò ban đầu mà không lưu thay đổi.
            \end{enumerate}
    \end{enumerate} \\
    \hline
    Các yêu cầu đặc biệt &
    \begin{itemize}
        \item Hỗ trợ gán quyền chi tiết cho từng module: sản phẩm, đơn hàng, danh mục, khách hàng, khuyến mãi...
        \item Cho phép sao chép hoặc nhân bản vai trò để tạo nhanh vai trò mới.
        \item Lưu lịch sử thay đổi quyền và người thực hiện.
        \item Áp dụng cơ chế RBAC (Role-Based Access Control) để đảm bảo an toàn truy cập.
    \end{itemize} \\
    \hline
\end{longtable}

