\chapter{Phân tích thiết kế hệ thống}

\section{Sơ đồ Use Case - Tác nhân và hành động}

\subsection{Sơ đồ Use Case tổng quát}
\begin{figure}[H]
    \centering
    \includegraphics[scale=0.4]{images/Sơ đồ usecase tổng quát.png}
    \caption{Sơ đồ Use Case tổng quát}
\end{figure}

\subsection{Usecase đăng ký}
\begin{figure}[H]
    \centering
    \includegraphics[scale=0.4]{images/usecase đăng ký.png}
    \caption{Sơ đồ usecase đăng ký}
\end{figure}
\begin{longtable}{|m{4cm}|m{10cm}|}
    \hline
    \textbf{Tên Use Case} & \textbf{Đăng ký} \\
    \hline
    Tác nhân & Khách hàng \\
    \hline
    Mô tả & Khách hàng sử dụng chức năng này để đăng ký tài khoản phục vụ cho đăng nhập vào hệ thống \\
    \hline
    Điều kiện kích hoạt & Chọn chức năng đăng ký ở hệ thống  \\
    \hline
    Tiền điệu kiện & Tác nhân chưa đăng nhập và chưa có tài khoản ở hệ thống \\
    \hline
    Hậu diều kiện & 
    \begin{itemize}
        \item Nếu thành công: Hệ thống sẽ thông báo đăng ký tài khoản thành công. Người dùng được chuyển hướng đén trang đăng nhập.
        \item Nếu thất bại: Hệ thống sẽ đưa ra thông báo chi tiết lỗi và yêu cầu nhập lại thông tin.
    \end{itemize} \\
    \hline
    Luồng sự kiện chính & 
    \begin{enumerate}[label=\arabic{*}.]
        \item Hệ thống hiển thị giao diện đăng nhập
        \item Người dùng bấm vào nút đăng ký
        \item Hệ thống hiển thị giao diện đăng ký
        \item Người dùng nhập tài khoản bao gồm email mật khẩu và xác nhận mật khẩu
        \item Hệ thống kiểm tra thông tin đăng ký
        \item Hiển thị giao diện xác nhận email
        \item Người dùng nhập mã otp được gửi về email
        \item Chuyển hướng đến trang đăng nhập
        \item Kết thúc usecase
    \end{enumerate} \\
    \hline
    Luồng sự kiện phụ & 
        \begin{enumerate}[label=\textbf{\arabic*.}, leftmargin=1.2em]
        \item \textbf{Người dùng hủy đăng ký}
            \begin{enumerate}[label*=\arabic*.]
            \item Người dùng chọn “Hủy” để rời trang đăng ký.
            \item Hệ thống hiển thị hộp thoại xác nhận hủy.
            \item Nếu xác nhận: hủy phiên đăng ký tạm, xóa dữ liệu chưa lưu, không tạo tài khoản.
            \item Hệ thống chuyển hướng về trang đăng nhập.
            \item Kết thúc usecase.
            \end{enumerate}
        \item \textbf{Email đã tồn tại} (phát sinh ở bước 5 luồng chính):
            \begin{enumerate}[label*=\arabic*.]
            \item Hệ thống báo “Email đã được sử dụng”.
            \item Yêu cầu người dùng đăng nhập hoặc chọn “Quên mật khẩu”.
            \item Kết thúc usecase.
            \end{enumerate}

        \item \textbf{Mã xác thực email hết hạn} (ở bước 7 luồng chính):
            \begin{enumerate}[label*=\arabic*.]
            \item Hiển thị thông báo “Mã hết hạn”.
            \item Cung cấp nút “Gửi lại mã”.
            \item Khi gửi lại thành công, quay lại luồng chính tại bước 6.
            \item Kết thúc usecase.
            \end{enumerate}
        \end{enumerate} \\
    \hline
    Các yêu cầu đặc biệt & Không có \\
    \hline
\end{longtable}
\subsection{Usecase đăng nhập}
\begin{figure}[H]
    \centering
    \includegraphics[scale=0.4]{images/usecase đăng nhập.png}
    \caption{Sơ đồ usecase đăng nhập}
\end{figure}

\begin{longtable}{|m{4cm}|m{10cm}|}
    \hline
    \textbf{Tên Use Case} & \textbf{Đăng nhập} \\
    \hline
    Tác nhân & Người dùng \\
    \hline
    Mô tả & Tác nhân sử dụng usecase này để đăng nhập vào hệ thống\\
    \hline
    Điều kiện kích hoạt & Chọn chức năng đăng nhập ở hệ thống  \\
    \hline
    Tiền điệu kiện & Tác nhân chưa đăng nhập vào hệ thống \\
    \hline
    Hậu diều kiện & 
    \begin{itemize}
        \item Nếu thành công: Hệ thống sẽ hiển thị giao diện chính. Người dùng có thể thực hiện các chức năng theo đúng quyền hạn của mình
        \item Nếu thất bại: Hệ thống sẽ đưa ra thông báo "Thông tin đăng nhập không hợp lệ" và yêu cầu đăng nhập lại.
    \end{itemize} \\
    \hline
    Luồng sự kiện chính & 
    \begin{enumerate}[label=\arabic{*}.]
        \item Hệ thống hiển thị giao diện đăng nhập.
        \item Người dùng nhập tài khoản bao gồm tên đăng nhập và mật khẩu.
        \item Hệ thống kiểm tra và xác nhận thông tin đăng nhập.
        \item Hiển thị giao diện chính tùy vào vai trò của tài khoản đăng nhập là người dùng hay admin.
        \item Kết thúc usecase.
    \end{enumerate} \\
    \hline
    Luồng sự kiện phụ & 
        \begin{enumerate}[label=\textbf{\arabic*.}, leftmargin=1.2em]
        \item \textbf{Thông tin email hoặc mật khẩu sai}
            \begin{enumerate}[label*=\arabic*.]
            \item Người dùng nhập thông tin sai
            \item Hệ thống báo lỗi
            \item Kết thúc Use Case
            \end{enumerate}
        \end{enumerate} \\
    \hline
    Các yêu cầu đặc biệt & Không có \\
    \hline
\end{longtable}

\subsection{Usecase đăng xuất}
\begin{figure}[H]
    \centering
    \includegraphics[scale=0.6]{images/usecase đăng xuất.png}
    \caption{Sơ đồ usecase đăng xuất}
\end{figure}

\begin{longtable}{|m{4cm}|m{10cm}|}
    \hline
    \textbf{Tên Use Case} & \textbf{Đăng xuất} \\
    \hline
    Tác nhân & Người dùng \\
    \hline
    Mô tả & Tác nhân sử dụng usecase này để đăng xuất khỏi hệ thống\\
    \hline
    Điều kiện kích hoạt & Chọn chức năng đăng xuất ở hệ thống  \\
    \hline
    Tiền điệu kiện & Tác nhân đã đăng nhập vào hệ thống \\
    \hline
    Hậu diều kiện & 
    \begin{itemize}
        \item Nếu thành công: Hệ thống sẽ hiển thị giao diện khi chưa đăng nhập.
        \item Nếu thất bại: Hệ thống sẽ đưa ra thông báo lỗi.
    \end{itemize} \\
    \hline
    Luồng sự kiện chính & 
    \begin{enumerate}[label=\arabic{*}.]
        \item Hệ thống hiển thị giao diện đăng xuất.
        \item Người dùng chọn đăng xuất.
        \item Hệ thống tiến hành đăng xuất tài khoản người dùng.
        \item Hiển thị giao diện khi chưa đăng nhập.
        \item Kết thúc usecase.
    \end{enumerate} \\
    \hline
    Luồng sự kiện phụ & 
        \begin{enumerate}[label=\textbf{\arabic*.}, leftmargin=1.2em]
        \item \textbf{Đăng xuát thất bại}
            \begin{enumerate}[label*=\arabic*.]
            \item Người dùng đăng xuất thất bại
            \item Hệ thống báo lỗi
            \item Kết thúc Use Case
            \end{enumerate}
        \end{enumerate} \\
    \hline
    Các yêu cầu đặc biệt & Không có \\
    \hline
\end{longtable}

\subsection{Usecase đổi mật khẩu}
\begin{figure}[H]
    \centering
    \includegraphics[scale=0.6]{images/usecase đổi mật khẩu.png}
    \caption{Sơ đồ usecase đổi mật khẩu}
\end{figure}

\begin{longtable}{|m{4cm}|m{10cm}|}
    \hline
    \textbf{Tên Use Case} & \textbf{Đổi mật khẩu} \\
    \hline
    Tác nhân & Người dùng \\
    \hline
    Mô tả & Tác nhân sử dụng usecase này để thay đổi mật khẩu của tài khoản\\
    \hline
    Điều kiện kích hoạt & Chọn chức năng đổi mật khẩu ở khu vực tài khoản  \\
    \hline
    Tiền điệu kiện & Tác nhân đã đăng nhập vào hệ thống \\
    \hline
    Hậu diều kiện & 
    \begin{itemize}
        \item Nếu thành công: Hệ thống sẽ đăng xuất tài khoản người dùng và chuyển hướng đến trang đăng nhập.
        \item Nếu thất bại: Hệ thống sẽ đưa ra thông báo lỗi.
    \end{itemize} \\
    \hline
    Luồng sự kiện chính & 
    \begin{enumerate}[label=\arabic{*}.]
        \item Hệ thống hiển thị giao diện quản lý tài khoản.
        \item Người dùng chọn đổi mật khẩu.
        \item Hệ thống hiển thị giao diện đổi mật khẩu.
        \item Người dùng nhập mật khẩu cũ, mật khẩu mới và xác nhận mật khẩu mới.
        \item Hệ thống kiểm tra và xác nhận thông tin vừa nhập.
        \item Hiển thị thông báo đổi mật khẩu thành công và lưu thông tin mật khẩu mới vào cơ sở dữ liệu.
        \item  Hệ thống đăng xuất người dùng và chuyển hướng đến trang đăng nhập.
        \item Kết thúc usecase.
    \end{enumerate} \\
    \hline
    Luồng sự kiện phụ & 
        \begin{enumerate}[label=\textbf{\arabic*.}, leftmargin=1.2em]
        \item \textbf{Người dùng hủy yêu cầu đổi mật khẩu}
            \begin{enumerate}[label*=\arabic*.]
            \item Người dùng hủy yêu cầu đổi mật khẩu.
            \item Hệ thống bỏ qua và trở về giao diện quản lý tài khoản
            \item Kết thúc Use Case
            \end{enumerate}
        \item \textbf{Thông tin mật khẩu không hợp lệ}
            \begin{enumerate}[label*=\arabic*.]
            \item Người dùng nhập thông tin mật khẩu không hợp lệ
            \item Hệ thống thông báo lỗi
            \item Kết thúc Use Case
            \end{enumerate}
        \end{enumerate} \\
    \hline
    Các yêu cầu đặc biệt & Không có \\
    \hline
\end{longtable}


\subsection{Usecase chỉnh sửa thông tin tài khoản}
\begin{figure}[H]
    \centering
    \includegraphics[scale=0.6]{images/usecase chỉnh sửa thông tin.png}
    \caption{Sơ đồ usecase chỉnh sửa thông tin}
\end{figure}

\begin{longtable}{|m{4cm}|m{10cm}|}
    \hline
    \textbf{Tên Use Case} & \textbf{Chỉnh sửa thông tin} \\
    \hline
    Tác nhân & Người dùng \\
    \hline
    Mô tả & Tác nhân sử dụng usecase này để thay đổi thông tin của tài khoản\\
    \hline
    Điều kiện kích hoạt & Chọn chức năng chỉnh sửa ở khu vực quản lý tài khoản  \\
    \hline
    Tiền điệu kiện & Tác nhân đã đăng nhập vào hệ thống \\
    \hline
    Hậu diều kiện & 
    \begin{itemize}
        \item Nếu thành công: Hệ thống sẽ thông báo thành công và chuyển hướng đến trang quản lý tài khoản.
        \item Nếu thất bại: Hệ thống sẽ đưa ra thông báo lỗi.
    \end{itemize} \\
    \hline
    Luồng sự kiện chính & 
    \begin{enumerate}[label=\arabic{*}.]
        \item Hệ thống hiển thị giao diện thông tin tài khoản hiện tại của người dùng.
        \item Người dùng chọn chức năng \textit{Chỉnh sửa thông tin}.
        \item Hệ thống hiển thị biểu mẫu (form) cho phép người dùng chỉnh sửa các trường thông tin như: họ tên, email, số điện thoại, địa chỉ, ảnh đại diện, v.v.
        \item Người dùng thay đổi các thông tin mong muốn và nhấn nút \textit{Lưu thay đổi}.
        \item Hệ thống kiểm tra tính hợp lệ của dữ liệu nhập (định dạng email, độ dài, ký tự hợp lệ...).
        \item Nếu dữ liệu hợp lệ, hệ thống cập nhật thông tin mới vào cơ sở dữ liệu.
        \item Hệ thống hiển thị thông báo \textit{Cập nhật thông tin thành công}.
        \item Hệ thống chuyển hướng người dùng về trang quản lý tài khoản với dữ liệu mới.
        \item Kết thúc Use Case.
    \end{enumerate} \\
    \hline
    Luồng sự kiện phụ & 
        \begin{enumerate}[label=\textbf{\arabic*.}, leftmargin=1.2em]
            \item \textbf{Người dùng hủy chỉnh sửa thông tin}
                \begin{enumerate}[label*=\arabic*.]
                    \item Trong quá trình chỉnh sửa, người dùng chọn \textit{Hủy}.
                    \item Hệ thống không lưu bất kỳ thay đổi nào và quay lại trang quản lý tài khoản.
                    \item Kết thúc Use Case.
                \end{enumerate}
            \item \textbf{Dữ liệu nhập không hợp lệ}
                \begin{enumerate}[label*=\arabic*.]
                    \item Người dùng nhập thông tin sai định dạng (ví dụ: email sai, số điện thoại không hợp lệ, để trống trường bắt buộc).
                    \item Hệ thống hiển thị thông báo lỗi cụ thể tại từng trường thông tin.
                    \item Người dùng điều chỉnh lại thông tin và gửi lại yêu cầu.
                \end{enumerate}
            \item \textbf{Email hoặc số điện thoại đã tồn tại}
                \begin{enumerate}[label*=\arabic*.]
                    \item Hệ thống phát hiện email hoặc số điện thoại trùng với tài khoản khác.
                    \item Hệ thống hiển thị thông báo “Email/Số điện thoại đã được sử dụng”.
                    \item Người dùng nhập giá trị khác hoặc hủy thao tác.
                \end{enumerate}
        \end{enumerate} \\
    \hline
    Các yêu cầu đặc biệt & 
    \begin{itemize}
        \item Các trường thông tin (email, số điện thoại) phải được kiểm tra định dạng và tính duy nhất trước khi lưu.
        \item Ảnh đại diện nếu có, phải được kiểm tra định dạng tệp và dung lượng tối đa.
        \item Hệ thống ghi log mọi thay đổi thông tin để phục vụ theo dõi và bảo mật.
        \item Các thay đổi quan trọng (email, số điện thoại) có thể yêu cầu xác minh qua mã OTP hoặc email xác nhận.
    \end{itemize} \\
    \hline
\end{longtable}


\subsection{Usecase tìm kiếm đối tượng}
\begin{figure}[H]
    \centering
    \includegraphics[scale=0.6]{images/usecase tìm kiếm đối tượng.png}
    \caption{Sơ đồ usecase tìm kiếm đối tượng}
\end{figure}

\begin{longtable}{|m{4cm}|m{10cm}|}
    \hline
    \textbf{Tên Use Case} & \textbf{tìm kiếm đối tượng} \\
    \hline
    Tác nhân & Người dùng \\
    \hline
    Mô tả & Tác nhân sử dụng usecase này để tìm kiếm thông tin trong hệ thống\\
    \hline
    Điều kiện kích hoạt & Chọn chức năng tìm kiếm ở hệ thống \\
    \hline
    Tiền điệu kiện & Tác nhân có quyền xem đối tượng \\
    \hline
    Hậu diều kiện & 
    \begin{itemize}
        \item Nếu thành công: Danh sách kết quả được hiển thị; khi chọn một mục sẽ chuyển hướng đến trang chi tiết hoặc kết quả tìm kiếm
        \item Nếu thất bại: không thay đổi dữ liệu. Hệ thống sẽ thông báo lỗi phù hợp
    \end{itemize} \\
    \hline
    Luồng sự kiện chính & 
    \begin{enumerate}[label=\arabic{*}.]
        \item Người dùng chọn vào ô tìm kiếm.
        \item Người dùng nhập thông tin tìm kiếm.
        \item Hệ thống hiển thị danh sách kết quả.
        \item Người dùng chọn một kết quả.
        \item Hệ thống chuyển hướng đến trang kết quả hoặc chi tiết.
        \item Kết thúc usecase.
    \end{enumerate} \\
    \hline
    Luồng sự kiện phụ & 
        \begin{enumerate}[label=\textbf{\arabic*.}, leftmargin=1.2em]
        \item \textbf{Không có kết quả}
            \begin{enumerate}[label*=\arabic*.]
                \item Hệ thống trả về danh sách rỗng.
                \item Hệ thống hiển thị thông tin danh sách tìm kiếm trống.
            \end{enumerate}
        \end{enumerate} \\
    \hline
    Các yêu cầu đặc biệt & Không có \\
    \hline
\end{longtable}


\subsection{Usecase xem chi tiết sản phẩm}
\begin{figure}[H]
    \centering
    \includegraphics[scale=0.6]{images/usecase xem chi tiết sản phẩm.png}
    \caption{Sơ đồ usecase xem chi tiết sản phẩm}
\end{figure}

\begin{longtable}{|m{4cm}|m{10cm}|}
    \hline
    \textbf{Tên Use Case} & \textbf{xem chi tiết sản phẩm} \\
    \hline
    Tác nhân & Khách hàng \\
    \hline
    Mô tả & Tác nhân sử dụng usecase này để xem chi tiết về thông tin của một sản phẩm cụ thể\\
    \hline
    Điều kiện kích hoạt & Tác nhân chọn một sản phẩm cụ thể để xem chi tiết  \\
    \hline
    Tiền điệu kiện & Tác nhân có quyền xem thông tin của sản phẩm \\
    \hline
    Hậu diều kiện & 
    \begin{itemize}
        \item Nếu thành công: Danh sách kết quả được hiển thị; khi chọn một mục sẽ chuyển hướng đến trang chi tiết hoặc kết quả tìm kiếm
        \item Nếu thất bại: không thay đổi dữ liệu. Hệ thống sẽ thông báo lỗi phù hợp
    \end{itemize} \\
    \hline
    Luồng sự kiện chính & 
    \begin{enumerate}[label=\arabic{*}.]
        \item Người dùng chọn vào nút xem chi tiết ở sản phẩm.
        \item Hệ thống hiển thị trang chi tiết sản phẩm.
        \item Kết thúc usecase.
    \end{enumerate} \\
    \hline
    Luồng sự kiện phụ & Không có \\
    \hline
    Các yêu cầu đặc biệt & Không có \\
    \hline
\end{longtable}

\subsection{Usecase Quản lý giỏ hàng}
\begin{figure}[H]
    \centering
    \includegraphics[scale=0.6]{images/usecase quản lí giỏ hàng.png}
    \caption{Sơ đồ Use Case quản lý giỏ hàng}
\end{figure}

\begin{longtable}{|m{4cm}|m{10cm}|}
    \hline
    \textbf{Tên Use Case} & \textbf{Quản lý giỏ hàng} \\
    \hline
    Tác nhân & Khách hàng \\
    \hline
    Mô tả & 
    Cho phép khách hàng thêm sản phẩm vào giỏ hàng, chỉnh sửa số lượng, xóa sản phẩm, và xem tổng giá trị đơn hàng trước khi tiến hành thanh toán. \\
    \hline
    Điều kiện kích hoạt & 
    Khách hàng truy cập vào giỏ hàng hoặc thực hiện hành động thêm sản phẩm vào giỏ hàng từ trang chi tiết sản phẩm. \\
    \hline
    Tiền điều kiện &
    \begin{itemize}
        \item Khách hàng đã đăng nhập.
        \item Hệ thống có dữ liệu sản phẩm hợp lệ.
    \end{itemize} \\
    \hline
    Hậu điều kiện &
    \begin{itemize}
        \item Nếu thành công: Giỏ hàng được cập nhật (thêm/xóa/sửa số lượng), hiển thị tổng tiền chính xác.
        \item Nếu thất bại: Giỏ hàng không thay đổi, hệ thống hiển thị thông báo lỗi phù hợp.
    \end{itemize} \\
    \hline
    Luồng sự kiện chính &
    \begin{enumerate}[label=\arabic*.]
        \item Khách hàng truy cập trang giỏ hàng hoặc chọn “Thêm vào giỏ hàng” từ trang chi tiết sản phẩm.
        \item Hệ thống kiểm tra sản phẩm có tồn tại và còn hàng hay không.
        \item Nếu hợp lệ, hệ thống thêm sản phẩm vào giỏ hàng hoặc cập nhật số lượng (nếu đã có).
        \item Khách hàng có thể:
        \begin{itemize}
            \item Chỉnh sửa số lượng từng sản phẩm.
            \item Xóa sản phẩm khỏi giỏ hàng.
            \item Xem tổng giá trị giỏ hàng (bao gồm thuế, giảm giá nếu có).
        \end{itemize}
        \item Hệ thống hiển thị giỏ hàng đã cập nhật và tổng tiền tạm tính.
        \item Khách hàng chọn “Thanh toán” để chuyển sang quy trình đặt hàng.
        \item Kết thúc Use Case.
    \end{enumerate} \\
    \hline
    Luồng sự kiện phụ &
    \begin{enumerate}[label=\textbf{\arabic*.}, leftmargin=1.2em]
        \item \textbf{Sản phẩm hết hàng}
            \begin{enumerate}[label*=\arabic*.]
                \item Khách hàng thêm sản phẩm nhưng hệ thống phát hiện đã hết hàng.
                \item Hệ thống hiển thị thông báo “Sản phẩm đã hết hàng” và không thêm vào giỏ.
            \end{enumerate}
        \item \textbf{Cập nhật số lượng không hợp lệ}
            \begin{enumerate}[label*=\arabic*.]
                \item Khách hàng nhập số lượng vượt quá tồn kho hoặc nhỏ hơn 1.
                \item Hệ thống hiển thị thông báo “Số lượng không hợp lệ” và giữ giá trị cũ.
            \end{enumerate}
        \item \textbf{Giỏ hàng trống}
            \begin{enumerate}[label*=\arabic*.]
                \item Khách hàng mở giỏ hàng nhưng chưa thêm sản phẩm nào.
                \item Hệ thống hiển thị thông báo “Giỏ hàng của bạn đang trống” và gợi ý xem sản phẩm.
            \end{enumerate}
        \item \textbf{Lỗi hệ thống hoặc mất kết nối}
            \begin{enumerate}[label*=\arabic*.]
                \item Khi thêm/xóa sản phẩm, kết nối đến máy chủ thất bại.
                \item Hệ thống hiển thị thông báo lỗi “Không thể cập nhật giỏ hàng, vui lòng thử lại sau”.
            \end{enumerate}
    \end{enumerate} \\
    \hline
    Các yêu cầu đặc biệt &
    \begin{itemize}
        \item Giỏ hàng cần được lưu tự động trong tài khoản khách hàng.
        \item Cập nhật tổng giá trị giỏ hàng theo thời gian thực khi thay đổi số lượng.
        \item Cho phép tiếp tục mua hàng hoặc chuyển đến thanh toán dễ dàng.
        \item Kiểm tra tồn kho và giá sản phẩm mỗi khi khách hàng mở lại giỏ hàng.
    \end{itemize} \\
    \hline
\end{longtable}



\subsection{Usecase Quản lý danh sách yêu thích}
\begin{figure}[H]
    \centering
    \includegraphics[scale=0.6]{images/usecase quản lí danh sách yêu thích.png}
    \caption{Sơ đồ Use Case quản lý danh sách yêu thích (dành cho khách hàng)}
\end{figure}

\begin{longtable}{|m{4cm}|m{10cm}|}
    \hline
    \textbf{Tên Use Case} & \textbf{Quản lý danh sách yêu thích} \\
    \hline
    Tác nhân & Khách hàng \\
    \hline
    Mô tả & 
    Cho phép khách hàng thêm hoặc xóa sản phẩm khỏi danh sách yêu thích, xem danh sách yêu thích của mình và chuyển sản phẩm từ yêu thích sang giỏ hàng. \\
    \hline
    Điều kiện kích hoạt &
    Khách hàng chọn hành động ``Thêm vào yêu thích'' trên trang chi tiết sản phẩm hoặc mở trang ``Danh sách yêu thích''. \\
    \hline
    Tiền điều kiện &
    \begin{itemize}
        \item Khách hàng đã đăng nhập vào hệ thống.
        \item Sản phẩm còn tồn tại và có thể yêu thích.
    \end{itemize} \\
    \hline
    Hậu điều kiện &
    \begin{itemize}
        \item Nếu thành công: Danh sách yêu thích được cập nhật và hiển thị chính xác trên giao diện của khách hàng.
        \item Nếu thất bại: Danh sách không thay đổi, hệ thống hiển thị thông báo lỗi.
    \end{itemize} \\
    \hline
    Luồng sự kiện chính &
    \begin{enumerate}[label=\arabic*.]
        \item Khách hàng đăng nhập và mở trang chi tiết sản phẩm.
        \item Khách hàng nhấn nút \textit{Thêm vào yêu thích}.
        \item Hệ thống kiểm tra trạng thái sản phẩm (đã có trong yêu thích hay chưa).
        \item Nếu chưa có, hệ thống thêm sản phẩm vào danh sách yêu thích của khách hàng.
        \item Hệ thống cập nhật giao diện.
        \item Khách hàng có thể mở trang danh sách yêu thích để xem các sản phẩm đã lưu.
        \item Tại đây, khách hàng có thể xóa sản phẩm khỏi yêu thích hoặc chuyển sang giỏ hàng.
        \item Hệ thống xử lý hành động tương ứng và hiển thị danh sách cập nhật.
        \item Kết thúc Use Case.
    \end{enumerate} \\
    \hline
    Luồng sự kiện phụ &
    \begin{enumerate}[label=\textbf{\arabic*.}, leftmargin=1.2em]
        \item \textbf{Khách hàng chưa đăng nhập}
            \begin{enumerate}[label*=\arabic*.]
                \item Khách hàng nhấn ``Thêm vào yêu thích'' khi chưa đăng nhập.
                \item Hệ thống chuyển hướng đến trang đăng nhập và yêu cầu đăng nhập trước khi thao tác.
                \item Kết thúc usecase
            \end{enumerate}
        \item \textbf{Sản phẩm đã có trong danh sách yêu thích}
            \begin{enumerate}[label*=\arabic*.]
                \item Hệ thống phát hiện sản phẩm đã nằm trong danh sách.
                \item Hệ thống cho phép bỏ yêu thích (xóa sản phẩm khỏi danh sách).
                \item Kết thúc usecase
            \end{enumerate}
        \item \textbf{Sản phẩm không còn khả dụng}
            \begin{enumerate}[label*=\arabic*.]
                \item Sản phẩm trong danh sách bị ngừng kinh doanh hoặc hết hàng.
                \item Hệ thống hiển thị thông báo và gợi ý xóa khỏi danh sách.
                \item Kết thúc usecase
            \end{enumerate}
    \end{enumerate} \\
    \hline
    Các yêu cầu đặc biệt &
    \begin{itemize}
        \item Danh sách yêu thích được lưu riêng cho từng tài khoản khách hàng.
        \item Trạng thái yêu thích được cập nhật theo thời gian thực.
        \item Hệ thống kiểm tra tồn kho khi hiển thị danh sách yêu thích.
        \item Hệ thống chỉ cho phép khách hàng đã đăng nhập thực hiện thao tác yêu thích.
    \end{itemize} \\
    \hline
\end{longtable}
