\chapter{Khảo sát}
\section{Tại sao phải xây dựng website bán hàng}
Hoạt động bán hàng chia làm 2 kệnh chính là kênh truyền thống(bán hàng tại cửa hàng vật lý, tại chợ, siêu thị…) và kênh hiện đại (bán hàng tại website, sàn thương mại điện tử, mạng xã hội, nhóm online…). Website sẽ cung cấp một kênh bán hàng online khác ngoài cửa hàng truyền thống. Xu hướng sử dụng internet và mua hàng online đang tăng cao tại Việt Nam và toàn thế giới. Website có thể tiếp cận lượng lớn khách hàng mới và gia tăng doanh số.
\indent Dưới đây là một số lợi ích khi xây dựng website thương mại điện tử:
\begin{itemize}
    \item \textbf{Tăng lợi nhuận:} Website thương mại điện tử với lợi thế không bị giới hạn về mặt thời gian và không gian như hình thức bán hàng truyền thống. Vì vậy nó có thể chủ động tìm kiếm khách hàng đến với sản phẩm, dịch vụ của mình và ngược lại. Bên cạnh đó, để lợi nhuận tăng cao thì doanh nghiệp phải cung cấp ra những sản phẩm có chất lượng đảm bảo, phù hợp với nhu cầu người dùng, mức giá hợp lý cùng với sự hỗ trợ nhiệt tình, càng ngày chắc chắn hoạt động kinh doanh của công ty bạn càng phát triển một cách bền vững.
    \item \textbf{Tiết kiệm chi phí:} Tiết kiệm chi phí chính là lợi ích thu hút nhiều doanh nghiệp đầu tư để sở hữu cho mình một website thương mại điện tử. Bởi chủ doanh nghiệp không cần phải bỏ ra một khoản tiền lớn để thuê mặt bằng, thuê và đào tạo đội ngũ nhân viên,... mà vẫn thu được lợi nhuận cao nếu như doanh nghiệp đó biết kết hợp với một số chiến lược Marketing online.
    \item \textbf{Gia tăng khả năng tương tác với khách hàng:} Khi khách hàng vào trang web thương mại điện tử, họ sẽ dễ dàng cập nhật được giá cả, thông tin sản phẩm và các dịch vụ. Bên cạnh đó, bất cứ lúc nào khách hàng cần tư vấn thì đội ngũ chăm sóc khách hàng sẽ tư vấn kịp thời, không để khách hàng phải chờ đợi lâu. Chính điều này sẽ thể hiện sự kính trọng, chuyên nghiệp và uy tín của công ty trong lòng khách hàng giúp tăng khả năng tương tác với khách.
    \item \textbf{Nâng cao tính cạnh tranh với đối thử:} Cuộc chiến cạnh tranh trên thị trường bán hàng trên công nghệ ngày càng khốc liệt, nhưng nếu doanh nghiệp biết tận dụng thời cơ này để sở hữu cho mình một website thương mại điện tử với giao diện dễ nhìn, độc đáo thì sẽ giúp thu hút nhiều khách hàng hơn. Vì thế, doanh nghiệp cần phải thiết lập một website thật độc đáo, mang đầy tính sáng tạo để khiến cho khách hàng ấn tượng, nhớ đến và lựa chọn sản phẩm, dịch vụ mà bạn đang kinh doanh.
    \item \textbf{Quảng bá thương hiệu:} Thời đại công nghệ số, người người nhà nhà đều sử dụng công nghệ kết nối Wifi/ Internet nên sẽ rất thuận lợi cho việc quảng bá thương hiệu không chỉ ở Việt Nam mà còn lan ra cả thị trường quốc tế. Từ đó, việc thiết kế trang web thương mại điện tử sẽ giúp bạn dễ dàng tiếp cận khách hàng tiềm năng ngày một nhanh chóng và hiệu quả hơn. 
\end{itemize}
\noindent  Có thể nói rằng, sở hữu một website bán hàng chính là điều cần thiết trong thời buổi cạnh tranh mang tính toàn cầu như hiện nay.

\section{Khảo sát hệ thống}
\indent Quy trình nghiệp vụ là một chuỗi các công việc theo thứ tự nhất định cần được thực hiện để đạt được những mục tiêu kinh doanh. Việc triển khai công việc theo quy trình nghiệp vụ sẽ đảm bảo các mục tiêu lớn nhỏ của doanh nghiệp được đáp ứng và không bị bỏ sót. Quy trình nghiệp vụ cũng thể hiện các thành phần chính quan trọng của quá trình kinh doanh của doanh nghiệp.

\indent Quy tình nghiệp vụ quản lí cửa hàng:
\begin{itemize}
    \item \textbf{Quy trình nhập hàng:} Cửa hàng cần cung cấp hàng hoá bằng cách mua từ những nhà sản xuất hoặc nhà cung cấp.
    \item \textbf{Quy trình lưu trữ và quản lí kho hàng:} Xác định vị trí và lưu trữ hàng hoá trong kho để dễ quản lí. Theo dõi số lượng tồn kho và cập nhật khi hàng khi có giao dịch nhập hoặc xuất hàng. Xác định sản phẩm sắp hết hàng và đặt hàng mới từ nhà cung cấp.
    \item \textbf{Quy trình bán hàng:} Tuỳ thuộc và mục đích cửa hàng sẽ có hai cách bán hàng là bán buôn và bán lẻ. Với mỗi cách sẽ có hai hình thức bán hàng là bán hàng offline và bán hàng online.
    \begin{itemize}
        \item Bán hàng offline: Quảng cáo chăm sóc bề ngoài của quán để gây ấn tượng và thu hút khách hàng. Khách hàng bước vào thì nhân viên chào đón. Khi khách hàng xem xét hoặc cân nhắc sản phẩm sẽ được nhân viên tư vấn ngắn gọn dễ hiểu về chi tiết sản phẩm. Khi khách hàng đã quyết định mua một sản phẩm và thực hiện thanh toán tiền, người Thu ngân sẽ là người thu tiền và áp dụng thêm một kĩ thuật upsale trong quá trình thu tiền. Thiết lập và duy trì mối quan hệ với khách hàng, sau khi khách hàng bước ra khỏi cửa hàng, thực hiện việc chăm sóc khách hàng như thế nào để nhằm thu hút khách hàng quay trở lại.
        \item Bán hàng online: Quản lí đăng các thông tin về sản phẩm lên bao gồm các thông tin như số lượng giá cả khuyến mãi, thông số. Sau khi người dùng cân nhắc và xem xét về sản phẩm, họ cần đăng nhập vào hệ thống để thêm sản phẩm vào giỏ hàng và hoàn thành thủ tục thanh toán. Khách hàng cung cấp thông tin về hình thức thanh toán, địa chỉ nhận hàng và số điện thoại liên hệ. Khi xác nhận thanh toán thành công thì cửa hàng sẽ giao sản phẩm cho đơn vị vận chuyển. Tuỳ vào từng khu vực mà phí vận chuyển sẽ khác nhau. Khách hàng bấm vào đã nhận hàng thành công để hoàn tất quá trình vận chuyển hàng hoá. Hệ thống cập nhật thông tin về đơn hàng và lưu trữ thông tin lại để phân tích cũng như chăm sóc khách hàng.
    \end{itemize}
    \item \textbf{Quy trình chăm sóc, quảng cáo và tiếp thị khách hàng:} 
    \begin{itemize}
        \item Xác định nhiệt độ của lưu lượng truy cập hoặc khách hàng hoặc tần suất đến cửa hàng: Kiểm tra tâm thế của lưu lượng truy cập trước khi đến trang của hệ thống. Lưu lượng truy cập có 3 mức độ: nóng, ấm, lạnh. Lưu lượng truy cập nóng chính là tập khách hàng đã từng mua hàng, đã trả tiền cho doanh nghiệp. Lưu lượng truy cập ấm chính là tập khách hàng đã trải nghiệm những sản phẩm quà tặng hoặc mua sản phẩm với giá thấp, hoặc là tập khách hàng của đối tác hoặc Affiliate. Lưu lượng truy cập lạnh chính là tập khách hàng mới, lần đầu tiên truy cập vào hệ thống hoặc mới đến với doanh nghiệp.
        \item Tạo cầu nối định hướng tâm thức khách hàng: Thông qua quảng cáo offline, trang Web, blog, email, kênh youtube, facebook, hoặc các mạng xã hội khác dưới dạng là bài viết hoặc video hoặc kết hợp dưới hình thức quảng cáo có trả phí hoặc không. Đối với tập khách hàng “nóng” thì đơn giản chỉ cần gửi email gắn với “link landingpage” cho khách hàng của bạn, bạn không cần phải giải thích nhiều về thông tin sản phẩm. Còn đối với lại khách hàng ấm thì bổ sung thêm thông tin ở trang landing page, nội dung dài hơn, hoặc tư vấn chi tiết hơn với nội dung gửi khách hàng nóng. Đối với khách hàng lạnh thì bắt buộc bạn phải có một cái trang Web cầu nối để giải thích ở trang Web này, phải giải thích toàn bộ những thuật ngữ lạ cho khách hàng trước khi chuyển khách hàng lạnh này từ trang “landingpage” sang trang “Sale page” để mua hàng hoặc tư vấn riêng chi tiết về các thông tin lợi ích giá cả cho khách hàng.
        \item Sàng lọc người đăng ký: Trao giá trị cho khách hàng và yêu cầu để lại thông tin cá nhân, nếu không cung cấp thì có thể lọc họ ra khỏi danh sách khách hàng tiềm năng.
        \item Sàng lọc người mua: Ngay khi khách hàng để lại thông tin cá nhân, phải dẫn họ đến trang mua hàng hoặc thực hiện tiếp cận khách hàng để giới thiệu sản phẩm.
        \item Nhận diện người mua mong muốn nhất thời: Người mua mong muốn nhất thời là những người đang cần sản phẩm ngay lập tức và sẽ mua nhiều món hàng cùng lúc nếu giúp họ có giải pháp ngay. Nếu không chỉ sau một thời gian khoảng một tuần, tâm lý không muốn mua là cao, khi biết kiểu khách hàng này thì cũng sẽ được chăm sóc theo cách riêng.
        \item Phát triển và nâng tầm mối quan hệ: Hãy chờ một thời gian, để họ tìm hiểu về sản phẩm đã mua và cho họ đủ thời gian để thấy giá trị mình mang đến. Từng bước bán họ những sản phẩm giá cao hơn.
        \item Thay đổi môi trường bán hàng: Thay đổi môi trường bán hàng là khi bán sản phẩm giá cao trên mạng. Thông qua việc gọi điện thoại, thư trực tiếp hoặc một sự kiện hay một hội thảo trực tiếp nào đó thì sẽ thuận lợi hơn để nhận được trực tiếp phản hồi thực tế của khách hàng và nhân viên bán hàng, giải quyết được lý do phản đối việc mua hàng và giúp khách hàng đưa ra quyết định nhanh chóng và khi thay đổi môi trường bán hàng thì chúng ta có thể giao tiếp ở một mức độ khác và việc đưa khách hàng lên mức cao cao hơn trong thang giá trị trở nên dễ dàng hơn.
    \end{itemize}
    \item \textbf{Quy trình báo cáo kinh doanh}: Tạo báo cáo về doanh thu, lợi nhuận, và hiệu suất cửa hàng để quản lí có thể đưa ra các chiến dịch và quyết định hợp lí.
    \begin{itemize}
        \item Đa số các doanh nghiệp quản lí doanh thu và nhân viên qua máy tính dựa vào các database hoặc qua sổ sách lưu trữ thông tin về ngày, ca làm việc, tên nhân viên, số tiền khi bắt đầu ca làm việc, số tiền khi kết thúc ca làm việc, số tiền chênh lệch.
        \item Quản lí cửa hàng là người duy nhất có quyền xem được báo cáo làm việc của nhân viên, tình hình thu nhập của doanh nghiệp, chỉnh sửa thông tin của sản phẩm. Nhân viên. Nhân viên chỉ xem được ca làm việc đó có bao nhiêu đơn hàng, đã hoàn thành bao nhiêu đơn hàng và tổng doanh thu của ca làm việc đó.
        \item Xác định các thông tin về khách hàng. Khi khách hàng đặt hàng phải xác định được thông tin về tên, địa chỉ, số điện thoại, thông tin về sản phẩm, thông tin về ngày thanh toán, phương thức thanh toán, cách thức vận chuyển, cước vận chuyển, tổng tiền phải trả.
    \end{itemize}
\end{itemize}

\indent Sau quá trình khảo sát các hệ thống và trang web cũng như các cửa hàng thì việc xây dựng trang web về thương mại điện tử cần:
\begin{itemize}
    \item Cung cấp thông tin chi tiết cho khách hàng về sản phẩm.
    \item Cung cấp các công cụ để chủ doanh nghiệp có thể quản lí một cách trực quan và đưa ra cách chăm sóc khách hàng hợp lí.
    \item Bảo mật thông tin, đem lại cho người dùng trải nghiệm tốt nhất có thể và tạo độ tin cậy cho khách hàng.
    \item Phân quyền ngườ dùng tỏng hệ thống của doanh nghiệp.
    \item Đáp ứng được các hình thức thanh toán đa dạng. Khi người dùng nhận được hàng và được xác nhận đã thanh toán thì đơn hàng lúc đó sẽ được đặt trạng thái thành công. Từ đó đưa ra các phân tích doanh thu chính xác.
\end{itemize}

\section{Phân tích yêu cầu}

\noindent\textbf{Đối với quản trị viên}:

\indent Hệ thống phải đáp ứng yêu cầu của quản trị viên như sau:
\begin{itemize}
    \item Chức năng đăng nhập vào hệ thống.
    \item Quản lí người dùng: Có quyền tạo các tài khoản nhân viên, khoá các tài khoản người dùng khi có hành động nguy hiểm.
    \item Quản lí đơn hàng: Chỉnh sửa đơn hàng, cập nhật trạng thái đơn hàng khi hoàn thành thanh toán.
    \item Quản lí đánh giá: Trả lời các bình luận của khách hàng, gửi email cho khách hàng về các chiến dịch quảng cáo hoặc khi sản phẩm khuyển mãi.
    \item Yêu cầu thống kê: Số lượng đơn hàng, người dùng đăng kì mới, số lượng sản phẩm, thu nhập theo thời gian.
\end{itemize}

\textbf{Đối với nhân viên}:

\indent Hệ thống phải đáp ứng yêu cầu của nhân viên như sau:
\begin{itemize}
    \item Chức năng đăng nhập cho nhân viên.
    \item Xem được báo cáo thống kê về doanh thu, số lượng và trạng thái đon hàng giờ làm việc.
    \item Trả lời các bình luận về sản phẩm.
    \item Tư vấn cho khách hàng qua email.
\end{itemize}
\textbf{Đối với khách hàng}:

\indent Hệ thống phải đáp ứng yêu cầu của khách hàng như sau:
\begin{itemize}
    \item Chức năng đăng nhập và đăng kí vào trang cho khách hàng.
    \item Hiển thị danh mục các sản phẩm. Các sản phẩm tiềm năng phải được đưa lên đầu và dễ nhìn với khách hàng. Khách hàng có thể xem thông tin chi tiết về thông số kỹ thuật của sản phẩm. Tìm kiếm sản phẩm qua từ khoá.
    \item Khi đăng nhập thành công, khách hàng có thể xem chi tiết trạng thái đơn hàng, lịch sử mua hàng, thông tin cá nhân của mình, thêm sản phẩm vào giỏ hàng.
    \item Tuỳ chọn hình thức thanh toán.
    \item Khi đã mua hàng khách hàng có thể đánh giá sản phẩm đã mua.
\end{itemize}